Las funciones constantes son trivialmente continuas.

\begin{define}
	Sean $X,Y$ espacios topologicos, es continua si $\forall U \subseteq Y $ abierto, $ f^{-1}(U)$ es abierto en $X$. 
\end{define}
La preimagen siempre preserva operaciones conjutistas. Basta checkar preimagenes de basicos sean abiertos, o incluso preimagenes de subbasicos sean abiertos. Fijar que la continuidad depende de las topologias.

Observar que: si meto mas abiertos en $X$ no daño la continuidad de la funcion. La topologia de $X$ mas fina no daña el checking de continuidad. Si miro en $Y$, si ellla tiene la topologia trivial cualquier funcion que llegue a $Y$ es continua. Por otro lado si $Y$ tiene la discreta es bien dificil. Si $X$ tiene la discreta, entonces cualquier funcion que salga de $X$ es continua.
\section{Topología Inicial}
Suponga $X$ es un conjunto, y tengo una familia $ (\set{Y}_{i\ \in I}, \tau_i) $ que son espacios topológicos y por cada $i$ tengo: $\function{f_i}{X}{Y_i}$  función. La topologia incial inducia en $X$ es la menor toplogia en $X$ para que todas las funciones $f_i$ sean continuas.

Esta topología es generada por estos conjuntos:
\[ \set{f_i^{-1} (u): u \in \tau_i, i \in I} \]

\textbf{Ejemplo 1: La Topología producto}
Sean $X,Y$ espacios topologicos, vamos a tomar de ahora en adelante la convencion $\tau_x, \tau_y$ para respectivas topologias.

Formen el producto cartesiano $ X \times Y$ y las funciones especiales son:
\begin{align*}
\function{\pi_1}{X \times Y}{X} \\
\function{\pi_2}{X \times Y}{X}
\end{align*}
 y esas dos las queremos continuas.
 
Entonces la topología generada por:

\[ \set{\pi_1^{-1} : u \in \tau_x} \cup \set{\pi_2^{-1} : v \in \tau_y} \]

Por lo menos tenemos que eso es una subbase. Rescritura:

\[ \set{U \times Y: u \in \tau_x} \cup \set{ X \times V : V \in \tau_y} \]

La interseccion es asociativa y es conmutativa. Observar que no es topologia pues union de cajas no es caja, pero es una base.
Pero basta meter los basicos(garantizar que las bases vayan a abiertos):

\textbf{Ejemplos del Ejemplo 1}
Típico: $ \RR \times \RR  = \RR^2 $
\textbf{Parentesis }
Sean $ \tau_1 $ y $\tau_2 $ y tengo sus bases $ \gothic{B_1} $ y $\gothic{B_2} $, observar que las condiciones simetricas $ \gothic{B_1} \contain \tau_2 \implies T_1 \contain \tau_2$. 
Coja un elemento de la base 1 e intercalar uno de la base 2. Y para probar lo otro, intercale.

En el plano, vemos que metemos cajas en discos, y discos entre cajas.

\textbf{Otro ejemplo: $ \mathfrak{N} \times \mathfrak{N} $}
Base para topologia de $ \mathfrak{N} $ es la discreta (cualquier base debe tener los singleton) y en $ \mathfrak{N} \times \mathfrak{N} = \mathfrak{N} $ y es de dimension cero, pues tiene una base de clopens, tambien tiene singleton \textbf{del producto cartesiano}.
Ojo $ \mathfrak{N} = \mathbb{N} $
 
\textbf{Otro ejemplo} Tomar $ (\RR_{\qt{discreta}} \times \RR_{\qt{usual}}) = \RR_{\qt{lex}}^2 $ Observar que entonces la topologia lexica no es tan exotica, viene de dos espacios metrizables.

\textbf{Ejercicio} Realizar y obtener una funcion de distancia para $\RR_{\qt{lex}} $

\textbf{Ejemplo 2} $(X, \tau)$ es un espacio topológico, $Y$ subconjunto de $X$, la funcion iteresante a observar es: inclusion:
\begin{align*}
i: Y &\rightarrow x \\
y&\rightarrowtail y
\end{align*}
el conjunto a observar (la topologia generada por):
\[ \set{i^{-1}(u) : u \in \tau} \]
observar que esto es igual
\[ u \cap Y : u \in \tau \]
Tenemos por lo menos que esto es una subbase y mirar que:
\begin{align*}
(U \cap Y) \cap (C \cap Y) \\
\end{align*}
esta colecion es cerrada bajo intersecciones finitas.
Que tal uniones:
\begin{align*}
( U \cap Y) \cup  (V \cap Y)\\
(U \cup V) \cap Y
\end{align*}
es decir que:
\[ \bigcup_{i \in I} (u_i \cap Y) = \lrp{ \bigcup_{i \in I} \cap Y} \]
Y vemos que esto ya es una topología porque ya hemos descritos todos los abierto.

Y si volvemos a la definicion original:
\[ \set{i^{-1}(u): u \in \tau} \]
y vemos que como preimagen respeta interseccion, de una observamos que es una topologia.
Concluimos que:
\[ \set{U \cap Y : U \in \tau} \]
es la topología de subespacio en $Y$ 
\textbf{Ejemplo}
$\RR$ con $ \mathbb{I} = \qt{irracionales} $, (intersecte los irracionales con los reales) son los abiertos.ahora mira que los imaginarios son cero dimensional. Observar que $ (p,q) \cap \mathbb{I} $ es una base de clopens. en terminologia los imaginarios son homemorfeos a los imaginarios cruz imaginarios. 
\textbf{Otro ejemplo: $ \RR_{\qt{lex}}^2 $ }
VER NOTAS, estamos considerando subconjunto y miramos las topologias dadas como subconjunto, o con la topologia del orden del suborden. 

Sean $X,Y$ espacios topologicos, $ X \times Y$ con la topologia del producto, y ahora tome $ A \times B \contain X \times Y$ (No todos los subconjuntos de $ X \times Y$ se puede escribir como cajas). Existen dos topologias que en buenas noticias son equivalentes. $ \tau_{\qt{subespacio del producto}}$ y  $ \tau_{q\qt{producto del subespacio}} $ . Por doble inclusión podemos ver:
Tome: $ U \in \tau_x $ y $ V \in \tau_y$ y vea que $ w = \lrp{A \times B} \cap \lrp{U \times V} $ y ahi observamos una de la inclusiones.

Para la segunda obsersevemos que:
\[ U \in \tau_x \quad U \cap A \quad V \in \tau_y \quad U \cap B \] y vemos entonces $ (U \cap A) \times (V \cap B) $ y vemos entonces como esas dos son equivalentes.