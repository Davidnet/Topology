Recordemos que en $ \RR^\omega $ ya hemos estudiado tres tipos de topologias en este conjunto:
\[ \tau_{\qt{prod}} \contain \tau_{\qt{unif}} \contain \tau_{\qt{cajas}} \]

\textbf{Teorema} $ X $ metrizable, $ A \contain X $,  $ p \in X $, $ p \in \bar{A}  \iff \exists(a_n) \contain A \quad a_n \rightarrow p$

\textbf{Definicion } $ X $ es Frechet-Uryson si $ \forall A \contain X  \forall p \in X$, $ p \in A \iff \exists(a_n) \contain A \quad a_n \rightarrow p $

y tenemos metrizable $ \implies $ primero contable $ \implies $ Frechet-Uryson

\textbf{Espacio de Arens} y \textbf{Espacio de Arens-Fohrs}

\href{https://dantopology.wordpress.com/2010/08/18/a-note-about-the-arens-space/	}{hypervinculo}

Tengo un espacio topologico $ X $ y una relación de $ \sim $, queremos ver la topología que puede tener $ x \overset{\rho}{\rightarrow} \frac{X}{\sim}$ y defino $ U \contain \frac{X}{\sim} $ U es abierto si y solo si $ f^{-1}(U) $ es abieto en $ X $. entonces $ \tau = \set{U \contain \frac{X}{\sim} | \rho^{-1}(U) \textrm{abierto en } X} $.
	
\textbf{Ejemplos}
Estudiar funciones cocientes.
\begin{define}
	$ \function{g}{X}{Y} $ es una aplicación cociente, si $ g^{-1}(U) $ es abierto en $X$ si y solo si $ U $ abierto en $Y$
\end{define}

\textbf{Arete Hawaiano} Es un subespacio de $ \RR^2 $