Recordemos entonces la definicion de espacio topologico.
\begin{define}
Dados $X$ conjunto y $ \tau \subset P(X) $ es un espacio topologico:
\begin{itemize}
	\item $X, \emptyset \in  \tau$
	\item $A \subset \tau \implies \cup A \in \tau $
	\item $ A \subset \tau \tau$ y $A$ es finito, implica que la intersecion finita esta en $ \tau $
\end{itemize}
\end{define}

\begin{exm}
	Si $ (X,d) $ es espacio métrico y $ \tau = \set{A \subseteq X | A \textrm{ es abierto}}$, entonce $ (X, \tau) $ es espacio topologico
\end{exm}
\begin{exm}
	Dado $X$, $ \tau_i = \set{\varnothing, X} $ es la topologia indiscreta trivial, o $ \tau_d = P(X) $ es la topologia discreta.
\end{exm}
\begin{exm}
	$ \Sigma $ es una teoría (Axiomas) de primer orden en el lenguaje $L$ (un ejemplo un simbolo de operacion binaria). Sea $X = \set{T | T \textrm{ teoría maximal consistente tal que } \Sigma \subset T} $. Sea $ \phi $ una sentencia (como soy abeliano), se armá un tipico abierto $ [ \phi ] = \set{T \in X | \phi \in T} $, observemos que $ X - [\phi] = \set{T \in X | \phi \not \in T} = [\textrm{NO } \phi]$ -Espacio de Stone-
\end{exm}
\begin{exm}
	Sea un campo $K$, y sea $X = k^n $, veamos la topología de Zariski, los cerrados son $ S \subseteq K[x_1, ..., x_n] $. los cerrados de s $ C_s = {\vec{x} \ in K^n | f(\vec{x})= 0 \forall f \in S} $. Todos los subconjuntos son compactos, 
\end{exm}
\begin{exm}
	Sea $X = \set{f: R \rightarrow R}$, un tipico abierto $ a \in R $, $ U \subseteq R$ abierto en la topología usual. Y armé el siguiente conjunto:
	\[ Y_{a,U} = \set{f \in X : f(a) \in U} \] Teoria de convergencia puntal.
\end{exm}
Veamos que aunque la union de topologias no es topologia,  dos topologias sobre un conjunto, se puede comparar.
tambien decimos $ \tau_1 \subset \tau 2$ decimos $\tau_1$ es mas gruesa y la otra es fina. Interseccion arbitrarias de topologias, es topologia (Probar!).
Podemos coger sea $X$ conjunto y $A \subset P(X) $:
\[ \bigcap\set{\tau \subset P(x) | \tau \textrm{es topología y } A \subset \tau} \] 
esta es la menor topologia que contiene a $A$ (la mas gruesa?). 
\begin{exm}
	ver notas
\end{exm}
\begin{define}
	Un punto aislado es cual el singleton de ese punto es abierto
\end{define}
\section{Como construir un a topologia}
\begin{define}
	$X$ conjunto, $B \contain P(x) $, $B$ es \textbf{base para una topologí}a sobre $X$ si
	\[ \set{\bigcup A | A \contain B} \]
	es topología
\end{define}
Esto no puede que no sea topologia, por dos razones os la union no es todo $X$, y que las interseciones finitas no son "abiertas"
Observar $ B \contain \set{\bigcup A | A \contain B} $.

\begin{define}
	\textbf{Definicion del libro} $X$ conjunto, $ B \contain P(X) $ es una base \ldots si:
	\[ \forall x \in X \exists b \in B x \in b. (\cup B = x) \]
	\[ \forall b_1, b_2 \in B \forall x \in b_1 \cap b_2 \exists b \in B \text{ such that } x \in b \contain b_1 \cap b_2 \]
\end{define}

\begin{thm}
	Las dos definiciones son equivalentes
\end{thm}

\begin{proof}
	$ 6 \implies 7 $  $ b_1, b_2 \in B$, $ x \in b_1 cap b_2$ (ver dibujos), como $ b_1, b_2$ esta en $ \tau$ la interseccion esta en $\tau$ (puede que no este en $B$), pero la interseccion (terminar)   
	
	$ 7 \implies 6 $
\end{proof}
¿Que quiere decir que un conjunto sea la union de un conjunto?

\begin{thm}
	Sea \[B = \set{B_\epsilon(x) = x \in X, \epsilon > 0 } \], probar que es una \textbf{topologia base}
\end{thm}
Las bolas en un espacio metricos es una base topologica.
\begin{define}
	Dado $ (X, \tau) $ un espacio toplogico:
	\begin{itemize}
		\item $B \contain \tau$ es base para $\tau $ si $ \forall U \in \tau \forall x \in U \exists b \in B $ t.q $ x \in B \contain U$
	\end{itemize}
\end{define}
Truco si estoy muy de buenas y $B$ es cerrado por intersecion condicion 2 es automaticamente ganada.
\begin{define}
	Un $S \contain P(X) $ es subbase si $ \bigcup S = X$
\end{define}
\begin{theorem}
	Si $S$ es subbase entonces $B = \set{ \bigcap A | A \contain S \text{ finitas }} $ es base para una topologia.
\end{theorem}
agregar intersecciones finitas y uniones arbitarias.
subbase genera topologia : coja un abierto y cheque que cualquier punto esta en la intersecion finita de alguno en la subbase.