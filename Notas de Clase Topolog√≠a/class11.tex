Primero contable implica Frechet -Uryhson.

En espacios metricos, todas las definiciones de compacidad, son las mismas. 

\textit{Lema} $ X $ secuencialmente compacto metrizable $ \implies $ Lema del número de Lebesgue.

Sea $ \mathcal{A} $ recubrimiento abierto de $ X $, $ \exists \delta > 0 $ tal que $  \forall B \subset X $ con $ \operatorname{diam}(B) < \delta \implies \exists A \in \mathcal{A} $ t.q $ B \subset A $. Por contradicción..

\textit{Lema} Vamos a ver que para todo $  \epsilon > 0 $ Existe, $ x_1, x_2, \ldots, x_n \in X $ tal que $ \cup_{i = 1}^n B_\epsilon(x_i) = X $

Con estos dos lemas.

\textbf{Teorema:} $ X $ metrico y secuencia compacto $ \implies $ compacidad
\textit{Demostración}
 $ X $ metrico y secuencialmente compacto $ \implies $ $ \mathcal{A} $ recubrimento de $ X $, sea  $  \delta > 0 $ número de Lebesgue de $ \mathcal{A} $
 
 \begin{define}
 	$ X $ es \textbf{localmente compacto} si $ \forall x \in X \exists K \subset X $ compacto tal que $ x \in \overset{\circ}{K} $
 \end{define}