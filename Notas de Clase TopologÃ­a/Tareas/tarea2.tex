\documentclass[notitlepage]{article}
\usepackage{lmodern}
\usepackage[T1]{fontenc}
\usepackage[spanish]{babel}
\usepackage[utf8]{inputenc}
\usepackage{amsmath}
\usepackage{amsfonts}
\usepackage{amssymb}
\usepackage{amsthm}
\usepackage{hyperref}
\author{David Cardozo}
\title{Tarea \# 2 (Ordenes, productos y subespacios)}
\newtheorem{define}{Definición}
\newtheorem{thm}{Teorema}
\newtheorem{prop}{Proposición}
\newtheorem{lem}{Lema}

%Customized Commands
\newcommand{\lrp}[1]{\left( #1 \right)}
\newcommand{\abs}[1]{\left| #1 \right|}
\newcommand{\set}[1]{\left\lbrace #1 \right\rbrace}
\newcommand{\RR}{\mathbb{R}}
\newcommand{\CC}{\mathbb{C}}
\newcommand{\QQ}{\mathbb{Q}}
\newcommand{\ZZ}{\mathbb{Z}}
\newcommand{\ZN}[1]{\frac{\mathbb{Z}}{#1 \mathbb{Z}}}
\newcommand{\PP}{\mathbb{P}}
\newcommand{\qt}[1]{\textrm{#1}}
\newcommand{\function}[3]{#1 : #2 \rightarrow #3}
\newcommand{\contained}{\subseteq}
\newcommand{\restric}[2]{ #1\restriction_{#2}}
\newcommand{\divs}{\mid}
\newcommand{\ndivs}{\nmid}

\begin{document}
\maketitle

\textbf{1.} Muestre que si $ (X, < )$ es un buen orden y $A \contained X$ tiene una cota superior entonces $A$ tiene supremo.


\textbf{2.}

%https://proofwiki.org/wiki/Closed_Set_in_Linearly_Ordered_Space

\textbf{3.}

\begin{thm}
	Sea $S_{\omega} $ con la topología del orden  y $ C \subseteq S_\omega $, $C $ es cerrados si y solo si para todo $A \contained C$, $ A \neq \varnothing$ enumerable, el supremo de $A$ esta en $C$
\end{thm}

\begin{proof}
	Primero, suponga $C$ cerrado, sea $A$ un conjunto no vacio de $C$, sea $b \in C^c$, queremos ver que $b$ no es un supremo de $A$.Observe que, si $b$ no es una cota superior de $A$, entonces por definición $b$ no puede ser un supremo. Suponga entonces que $b$ es una cota superior de $A$. Como $C$ es cerrado y $b \notin C$, debe existir un rayo abierto $U$ que contiene a $b$. Como $b$ es una cota superior de $A$ y $A \neq \varnothing$, $U$ no puede ser un rayo abierto de la forma $\set{x | x \leq b < a} $ para algún $a$. Por lo tanto $U$ es o un abierto o un rayo abierto dela forma $ \set{x | x > a} $. Tenemos entonces $ a < b$. Como $b$ es una cota superior de $A$, no elemento estrictamente mayor de elementos en $U$ puede estar en $S$, ahora $ A \contained C$, y el rayo $\set{x : x > a}$ la intersección con $S$ es vacía, concluimos que $a$ es una cota superior de $S$, como tenemos $ a < b$, $b$ no es un supremo de $S$. por lo tanto hemos mostrado que si $b$ esta en $C^c$, $b$ no es el supremo de $S$.
	
	 Segundo, suponga ahora que para ningún subconjunto de $C$, no hay supremo que este en $ C^c$, sea $ p \in C^c$, pueden ocurrir dos casos, o $p$ es una cota superior de $C$, o $p$ es una cota inferior, ambos casos son similares, entonces para el caso en que $p$ sea cota superior, como $C$ is un subconjunto de si mismo, sabemos entonces que no tiene un supremo en $C^c$, por lo tanto $p$ no es un supremo, asimismo existe entonces un $l \in S_\omega$ tal que $ l < p$, y vemos que el rayo $\set{x | x>a} $ tiene como miembros a $p$ y $ C \cap \set{x | x>a} = \varnothing $, y vemos que para cualquier abierto que contenga $p$ no contiene puntos de $C$. para el caso $p$ sea una cota inferior de $C$ el argumento es similar, y concluimos entonces que $C$ es un complemento de abiertos, es decir $C$ es cerrado, probando el si y solo si.
	 
\end{proof}


\textbf{4.} Si $L$ es una recta en el plano, describa la topología que $L$ hereda como subespacio de $\RR_l \times \RR$ y como subespacio de $\RR_l \times \RR_l $. En ambos casos se trata de una topología conocida.

\textbf{Solución}
Observar que una base para $\RR_l \times \RR$ es una colección de conjuntos de la forma $[a,b) \times (c,d)$. Por otra parte, una base para $\RR_l \times \RR_l$ es una colección de la forma $[a,b) \times [c,d)$. Observemos que la topología de la linea es de diferentes casos dependiendo en la pendiente de la linea, si la linea es vertical tenemos que para $\RR_l \times \RR$ la linea tiene la topología usual, mientras que la linea vertical... http://dbfin.com/topology/munkres/chapter-2/section-16-the-subspace-topology/?t=problem-8-solution



\begin{tabular}{|c|c|c|}
	\hline Dirección de la linea (con pendiente)  & $\RR_l \times \RR$   & $\RR_l \times \RR_l$  \\ 
	\hline $\uparrow$ & $\RR$  & $\RR_l$  \\ 
	\hline $\nearrow$ & $\RR_l$ & $\RR_l$ \\ 
	\hline $\rightarrow $ & $\RR_l$ & $\RR_l$  \\ 
	\hline $\searrow $ & $\RR_l$ & $\RR_d$  \\ 
	\hline $\downarrow $ & $\RR$ & $\RR_u$  \\ 
	\hline $\swarrow $ & $\RR_u$ & $\RR_u$  \\ 
	\hline $ \leftarrow $ & $\RR_u$ & $\RR_u$ \\ 
	\hline $\nwarrow $ & $\RR_u$ & $\RR_d$ \\ 
	\hline 
\end{tabular} 


\textbf{10}

\emph{(i)} Topología del producto y topología de diccionario.

Sea $ \tau_{\qt{prod}} $ la topología del producto, $\tau_{d} $ la topología de diccionario, y sea $ \tau_s$ la topología dada como subespacio del orden de diccionario. 

\begin{lem}
	$ \tau_{\qt{prod}} $ y $\tau_d$ no son comprables.
\end{lem}
\begin{proof}
	Primero, observese uqe $(0,1) $ pertenece al producto de $ [0,1] \times (\frac{1}{2},1] $ y este claramente no tiene ningún conjunto abierto que contenga  a esto en la topología de diccionario. Por otro lado, tenemos que $(0,\frac{1}{2}) \in \set{0} \times (0,1) $ y por lo tanto observamos no existe conjunto abierto que lo contenga en la primera topología, concluimos entonces que son no comparables.
\end{proof}
Antes de comparar la tercera topología (la de subespacio ) téngase en cuenta el siguiente lema.
\begin{lem}
	La topología dada por el orden de diccionario sobre $ \RR \times \RR $ es la misma topología dada por $ \RR_d \times \RR$ ($\RR_d$) es la topología discreta. 
\end{lem}
\begin{proof}
	Primero observemos que, a priori, la topología de orden de diccionario de $ \RR \times \RR$ es mas gruesa que la de  $ \RR_d \times \RR$, ya que cualquier intervalo en la topología de diccionario es la unión de abiertos en $ \RR_d \times \RR$. Ahora una base para $ \RR_d \times \RR$ es:
	\[ \set{\set{a} \times (b,c) | b<c} \] y vemos entonces que $$ \RR_d \times \RR$$ es mas fina que orden de diccionario sobre $ \RR \times \RR $, observamos también que esta topología (ya vimos que son la misma) es estrictamente mas fina que $ \RR^2$
\end{proof}

Ahora, volviendo al problema en general, la topología de producto es generada por (utilizando la información de la prueba anterior):
\[ \set{x} \times ((a,b) \cap [0,1]) \]
y vemos que cualquier elemento base de la primera topología, i.e, $ ((a,b) \cap [0,1]) \times ((c,d) \cap [0,1]) $ puede ser generada como unión de abiertos en la tercera topología, al mismo tiempo, vemos que cualquier base en la segunda topología, de la forma $(a,b) < (x,y) < (c,d) $ se puede generar como unión de abiertos, por ultimo utilizamos el Lema 1.3 del libro para ver que la tercera topología es estrictamente mas fina, por que las dos primeras son incomparables. En conclusión:

http://www.math.wsu.edu/faculty/remaley/525fa12hw2sol.pdf

\end{document}
