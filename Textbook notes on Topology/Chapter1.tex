\chapter{Set Theory and Logic}
\section{Fundamental Concepts}
We express that an object $a$ belongs to a set $A$ by the notation:
\begin{align*}
a \in A
\end{align*}
Similarly,
\[ a \not\in A \]
We denote the inclusion of a set into another set with:
\[ A \subseteq B \]
so that $ A = B  \iff A \subset B \textrm{and} B \subset A $. If $A \subset B$ but A is different from A, we say $A$ is a \textbf{proper subset} of $B$, in notation:
\begin{align*}
A \subsetneq B
\end{align*}
The relation $ \subset $ is called \textbf{inclusion} and $ \subsetneq $ is called proper inclusion. 
\subsection*{The Union of Sets and the Meaning of "or"}

Given two sets $A$ and $B$, we can form another set called the union of $A$ and $B$.

\[ A \cup B = \set{ x | x \in A \textrm{or } x\in B} \]
We will use the concept of exclusive or, if the necessity arises.

\subsection*{The Intersection of Sets, the Empty set, and the Meaning of "If...Then"}

Another way to form a set from two existing sets is to take the elements in common, that is:
\[ A \cap B = \set{ x | x \in A \textrm{ and } x \in B} \]

The \textbf{empty set } is the set with no elements, denoted by $ \emptyset $. We say that two elements are disjoint if: 
\[ A \cap B = \emptyset \]

Some property of this interesting empty set are:

\[ A \cap \emptyset = \emptyset \quad \quad A \cup \emptyset = A \]

 

