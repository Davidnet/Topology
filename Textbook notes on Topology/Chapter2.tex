\chapter{Topological Spaces and Continuous Functions}
\section{Topological Spaces}
\begin{define}
	A \textbf{topology} on a set $X$ is a collection $\tau$ of subsets of $X$ having the sollowing properties:
	\begin{itemize}
		\item $\emptyset \textrm{ and } X $ are in $\tau$ 
		\item The union of the elements of any subcollection of $ \tau $ is in $ \tau $
		\item The intersection of the elements of any finite subcollection of $ \tau $ is in $ \tau $.
	\end{itemize}
	A set $ X $ for which a topology $ \tau $ has been specified is called a \textbf{topological space}
\end{define}
Properly, a topological space is an ordered pair $ (X, \tau) $.

If $Z$ is a topological space with topology $ \tau $, we say that a subset $U$ of $X$ is an \textbf{open set} of $X$ if $U$ belongs to the collection $ \tau $. 

\begin{exm}
	If $X$ is any set, the collection of all subsets of $X$ is a topology on $X$; it is called the \textbf{discrete topology}. The collection consisting of $X$ and $\emptyset$ only is also a topology on $X$; we shall call it the \textbf{indiscrete topology}, or the \textbf{trivial topology}
\end{exm}
\begin{exm}
	Let $X$ be a set; let $\tau_f $ be the collection of all subsets of $U$ of $X$ such that $ X-U$ is either finite or is all of $X$. Then $\tau_f$ is a topology on $X$, called the \textbf{finite complement topology}. Both $X$ and $\varnothing$ are in $ \tau_f$, since $ X-X $ is finite and $ X - \varnothing $ is all of $X$. If $ \set{U_\alpha} $ is an indexed family of nonempty elements of $ \tau_f $, to show that $ \cup U_\alpha $ is in $\tau_f$, we compute
	\begin{align*}
	(\bigcup U_\alpha)^c = \bigcap U_{\alpha}^c
	\end{align*}
	and since each $U^c$ is finite, the union of these set is finite. If $ U_1, \ldots, U_n$ are nonempty elements of $ \tau_f$, to show that $ \cap U_i $ is in $ \tau_f$, er compute:
	\begin{align*}
		\lrp{\bigcap_{i=1}^n}^c U_i = \bigcup_{i=1}^n U_i^c
	\end{align*}
	Observe then that each $U_i^c $ is finite, and finite union of finite set is finite.
\end{exm}

\begin{exm}
		Let $X$ be a set; let $ \tau_c $ be the collection of all subsets $U$ of $X$ such that $ X - U$ either is countable or is all of $X$. Then $\tau_c$ is a topology on X.
\end{exm}

\begin{define}
	Suppose that $ \tau $ and $ \tau' $ are two topologies on a given set $X$. If $ \tau' \supset \tau $, we say that $ \tau' $ is \textbf{finer} than $ \tau$; if $\tau' $ properly contains $ \tau $, we say $ \tau'$ is \textbf{strictly finer} than $ \tau$. We also say that $ \tau $ is \textbf{coarser} than$ \tau' $, or \textbf{strictly coarser}, in these two respective situations. We say $ \tau $ is \textbf{comparable} with $ \tau' $ if either $ \tau' \supset \tau $ or $ \tau \supset \tau' $
\end{define}
\section*{Basis for a Topology}

\begin{define}
	If $X$ is a set, a \textbf{basis } for a topology on $X$ is a collection $ \gothic{B}$ of subsets of $X$ (called \textbf{basis elements}) such that:
	\begin{itemize}
		\item For each $x \in X $, there is at least one basis element $B$ containing $x$.
		\item If $x$ belongs to the intersection of two basis elements $B_1$ and $B_2$, then there is a basis element $B_3$ containing $x$ such that $ B_3 \subset B_1 \cap B_2 $.
	\end{itemize}
	If $\mathfrak{B}$ satisfies these two conditions, then we define the \textbf{topology $\tau$ generated by $\gothic{B}$} as follows: A subset $U$ of $X$ is said to be open in $X$ (that is , to be an element of $ \tau$) if for each $x \in U$, there is a basis element $B \in \gothic{B}$ such that $ x \in B$ and $ B \subset U$. Note that each basis element is itself an element of $ \tau $
\end{define}
\begin{exm}
	If $X$ is any set, the collection of all one-point subsets of $X$ is a basis for the discrete topology on $X$.
\end{exm}

\begin{lem}
	The collection $ \tau $ generated by the basis $ \gothic{B}$ is, in fact a topology on $X$
\end{lem}
\begin{proof}
	If $U$ is the empty set, it satisfies the defining condition of openness vacuously. Likewise, $X$ is in $\tau$, since for each $ x \in X$ there is some basis element $B$ containing $x$ and contained in $X$. Now let us take an indexed family $ \set{U_\alpha}_{\alpha \in J} $ of elements of $ \tau $, that is the collection $ \contained$ in $ \tau$ and show that:
	\[ U = \bigcup_{\alpha \in J} U_\alpha \in \tau \]
	Given $x \in U$, there exist an index $ \alpha $ such that $ x \in U_{\alpha} $. Since $ U_{\alpha} $ is open, there is a basis element $B$ such that $x \in B \contained U_{\alpha}$. Then $x \in B$ and $ B \contained U$, so that $U$ is open, by definition.
	Now let us take two elements $ U_1 $ and $ U_2$ of $ \tau $ and show that the intersection belongs to $ \tau $. Given $x \in U_1 \cap U_2 $, choose a basis element $B_1$ containing x such that $ B_1 \contained U_1$; choose also a basis $B_2$ containing $x$ such that $ B_2 \contained U_2$. The second condition for a basis enables us to choose a basis element $ B_3 $ containing x such that $ B_3 \contained B_1 \cap B_2$. Then $ x \in B_3$ and $ B_3 \contained U_1 \cap U_2 $, so $ U_1 \cap U_2 $ belongs to $ \tau $. Finally, we show by induction that any finite intersection $ U_! \cap \ldots \cap U_n$ of elements in $ \tau $ is in$ \tau$. The fact is trivial for $ n =1$; we suppose it true for $ n-1$ and prove it for $n$. Now
	\[ \lrp{U_1 \cap \ldots U_n } = \lrp{U_1 \cap \ldots U_{n-1} } \cap U_n \].
	By hypothesis, $ U_1 \cap \ldots U_{n-1} $ belongs to $ \tau $; by the result proven above, the intersection of $ U_1 \cap \ldots U_{n-1} $ and $U_n$ also belongs to $ \tau $
\end{proof}

Another point of view to view a basis for a topology is summarized in the following lemma:

\begin{lem}
	Let $X$ be a set; let $ \gothic{B} $ be a basis for a topology $ \tau $ on $X$. Then $ \tau $ equals the collection of all unions of elements of $ \gothic{B} $
\end{lem}

\begin{proof}
	By double inclusion, Given a collection of elements in $ \gothic{B} $, they are also elements of $ \tau $. Because $ \tau $ is a topology, their union is in $\tau$. Conversely, given $ U \in \tau $, choose for each $ x \in U$ an element $ B_x $ of $ \gothic{B} $ such that $ x \in B_x \contained U $. Then $ U = \bigcup_{x \in U} B_x$, so $U$ equals a union of elements of $ \gothic{B} $
\end{proof}
As an important observation, this says that every open set $U$ in $X$ can be written as a union of basis elements. This expression for $U$ is not unique. In summary, we have just described two ways from going from the basis to the topology it generates, now for the other way around:

\begin{lem}
	Let $X$ be a topological space. Suppose that $\gothic{C}$ is a collection of open sets of $X$ such that for each open set $U$ of $X$ and each $x $ in $U$, there is an element $C$ of $ \gothic{C} $ such that $ x \in C \contained U$. Then $ \gothic{C} $ is a basis for the topology of $X$.
\end{lem}

\begin{proof}
	We must show that $ \gothic{C} $ is a basis. The first condition for a basis is easy: Given $x \in X$, since $X$ is itself an open set, there is by hypothesis an element $C$ of $ \gothic{C} $ such that $ x \in C \contained X$. To check the second condition, let $x$ belong to $ C_1 \cap C_2$, where $C_1$ and $C_2$ are elements of $ \gothic{C} $. Since $C_! $ and $C_2$ are open, so is $ C_1 \cap C_2$. Therefore, there exists by hypothesis an element $C_3$in $ \gothic{C } $ such that $x \in C_3 \contained C_1 \cap C_2$.
	
	Let $ \tau $ be the collection of open sets of $X$; we must show that the topology $ \tau'$ generated by $ \gothic{C} $ equals the topology $ \tau $. First, note that if $U$ belongs to $\tau $ and if $ x \in U$ then there is by hypothesis an element $C$ of $ \gothic{C} $ such that $x \in C \contained U$. It follows that $U$ belongs to the topology $ \tau'$, by definition. Conversely, if $W$ belongs to the topology $ \tau'$, then $W$ equals a union of elements of $ \gothic{C} $, by the preceding lemma. Since each element of $\gothic{C}$ belongs to $ \tau $ and $ \tau $ is a topology. $W$ also belongs to $ \tau $	
\end{proof}

When topologies are given by bases, we want to have a criterion for deciding which topology is finer than other. The following provides a criterion:

\begin{lem}
	Let $\gothic{B} $ and $ \gothic{B'} $ be bases for the topologies $ \tau $ and $ \tau'$, respectively, on $X$. Then the following are equivalent:
	\begin{itemize}
		\item $ \tau' $ is finer than $ \tau $, i.e., $\tau' \supset \tau$
		\item For each $ x \in X$ and each basis element $B \in \gothic{B} $ containing $x$, there is a basis element $ B' \in \gothic{B'} $ such that $ x \in B' \contained B$.
	\end{itemize}
\end{lem}
\begin{proof}
	$" 2 \implies 1 " $. Given an element $U$ of $ \tau $, we wish to show that $ U \in \tau'$. Let $ x \in U $. Since $ \gothic{B} $ generates $\tau$, there is an element $ B \in \gothic{B} $ such that $ x \in B \contained U$. Condition (2) tell us there exists an element $ B' \in \gothic{B'} $ such that $x \in B'\contained B$. Then $x \in B'\contained U$, so $U \in \tau'$, by definition.
	
	$" 1 \implies 2 " $. We are given $ x \in X $ and $B \in \gothic{B} $, with $x \in B$. Now $B$ belongs to $ \tau $ by definition and $ \tau \contained \tau'$ by condition (1); therefore, $ B \in \tau'$. Since $ \tau'$ is generated by $ \gothic{B'}$, there is an element $ B'\in \gothic{B'}$ such that $ x \in B' \contained B$
\end{proof}


