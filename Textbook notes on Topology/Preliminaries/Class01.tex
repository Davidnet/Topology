\begin{define}
	A \textbf{rule of assignment} is a subset $r$ of the cartesian product $C \times D$ of two sets, having the property that each element of $C$ appears as the first coordinate of at most one ordered pair belonging to $r$
\end{define}
An equivalent formulation is
\[ [ (c,d) \in r \qt{ and } (c,d') \in r ] \implies [d = d'] \]

\begin{define}
	a \textbf{function } $f$ is a rule of assignment $r$, together with a set $B$ that contains the image set of $r$ . The domain $A$ of the rule $r$ is also called the \textbf{domain} of the function $f$; the image set of $r$ is also called the \textbf{image set} of $f$; and the set $b$ is called the \textbf{range} of $f$
\end{define}

\begin{define}
	If $\function{f}{A}{B}$ and if $A_0$ is a subset og $A$, we define the \textbf{restriction} of $f$ to $A_0$ to be the function mapping $A_0$ into $ B $ whose rule is:
	\[ \set{(a,f(a)) | a \in A_0} \]
	It is denoted by $ \restric{f}{A_0} $.
\end{define}

\begin{define}
	Given functions $\function{f}{A}{B}$ and $\function{g}{B}{C} $, we define the \textbf{composite} $ g \circ f$ of $f$ and $g$ as the function $ \function{g \circ f}{A}{C} $ defined by the equation $ (g \circ f)(a) = g(f(a)) $.
	Formally, $  g \circ f: A \rightarrow C $ is the function whose rule is:
	\[ \set{(a,c) | \qt{For some } b \in B, f(a)= b \qt{ and } g(b) = c} \]
\end{define}
We take note that $ g \circ f$ is defined only when the range of $f$ equals the domain of $g$
\begin{define}
	A function $ \function{f}{A}{B} $ is said to be \textbf{injective} (or \textbf{one-to-one}) if for each pair of distinct points of $A$, their images under $f$ are distinct. It is said to be \textbf{surjective} (or $f$ is said to map $A$ \textbf{onto} $B$) if every element of $B$ is the image of some element of $A$ under the function $f$, If $f$ is both injective and surjective, it is said to be \textbf{bijective} 
\end{define}
An important remark of facts is that, the composite of two surjective functions is surjective, and the composite of two injective functions is injective.

If $f$ is bijective there exists a function from $B$ to $A$ called the \textbf{inverse} of $f$. It is denoted by $ f^{-1} $ and is defined by letting $ f\inv(b) $ be that unique element $a$ of $A$   for which $f(a) = b$. 

\begin{lem}
	Let $\function{f}{A}{B} $. If there are functions $ \function{g}{B}{A} $ and $ \function{h}{B}{A} $ such that $ g(f(a)) = a $ for every $a$ in $A$, and $f(h(b)) = b$ for every $b$ in $B$, then $f$ is bijective and $ g = h= f\inv $
\end{lem}
\begin{proof}
	To prove that $f$ is surjective, observe that the function $h$ that goes from $B$ to $A$ maps every element to $h(b) \in A$ so that, every element of $b$ has an element of $A$ such that $ f(h(b)) = b$. To prove that it is injective, suppose for the sake of contradiction that there exists elements of $A$ such that $ a \neq a' $ but $f(a) = f(a') $ then, applying $g$ to both sides of this equation we have that $ g(f(a)) = g(f(a')) $ and by the conditions of the hypothesis we have that $a = a'$ which is a contradiction. To see that both are equal observe that:
	\begin{align*}
	h(b) &= g(f(h(b))) \\
		 &= g(b)		
	\end{align*}
	and the proof is complete.
\end{proof}
\begin{define}
	Let $ \function{f}{A}{B} $. If $A_0$ is a subset of $A$, we denote by $f(A_0)$ the set of all images of points of $A_0$ under the function $f$; this set is called the \textbf{image} of $A_0$ under $f$. Formally,
	\[ f(A_0) = \set{b | b = f(a) \qt{ for at least one } a \in A_0} \]
	On the other hand, if $ B_0 $  is a subset of $B$, we denote by $f\inv(B_0) $ the set of all elements of $A$ whose images under $f$ lie in $B_0$; it is called the \textbf{preimage} of $B_0$ under $f$ (or the ``counterimage'' or the ``inverse image'' of $B_0$). Formally,
	\[ f\inv(B_0) = \set{a | f(a) \in B_0} \]
	Of course, there may be no points $a$of $A$ whose images lie in $ B_0 $; in that case, $f\inv (B_0) $ is empty.
	
\end{define}

We make the following remark, and that is that: If $ \function{f}{A}{B} $  and if $A_0 \contained A$ and $B_0 \contained B$, then:
\[ A_0 \contained f\inv(f(A_0)) \quad \qt{and} \quad f(f\recip(B_0)) \contained B_0 \]

The first inclusion is an equality if $f$ is injective, and the second inclusion is an equality if $f$ is surjective.

\textbf{Relations}

\begin{define}
	A \textbf{relation} on a set $A$ is a subset $C$ of the Cartesian product $A \times A$ 
\end{define}

\textbf{Equuivalence Relations and Partitions}
An \textbf{equivalence relation} on a set $A$ is a relation $C$ on $A$ having the following three properties:
\begin{enumerate}
	\item (Reflexivity) $xCx$ for every $ x \in A $
	\item (Symmetry) If $ xCy $, then $ yCx $ 
	\item (Transitivity) If $ xCy $ and $ yCz $, then $ xCz $
\end{enumerate}  

Given an equivalent realtion $ \sim $ on a set $A$ and an element $x$ of $A$, we define a certain subset $E$ of $A$, called the \textbf{equivalence class} determined by $x$, via:
\[ E = \set{y | y \sim x} \]

\begin{lem}
	Two equivalence classes $E$ and $E'$ are either disjoint or equal
\end{lem}

\begin{proof}
	Let $E$ be the equivalence class determined by $x$, and let $E'$ be the equivalence class determined by $x'$. Suppose that $ E \cap E'$ is not empty; let $y$ be a point of $ E \cap E'$, observe that we have $ y ~ x $ and $ y ~ x'$ so  that we conclude that $ x ~ x'$. If now $w$ is any point of $E$, we have $w ~ x$ so that we conclude $E \contained E'$, by symmetry of the argument we see that the equality holds.
\end{proof}

\begin{define}
	A \textbf{partition} of a set $A$ is a collection of disjoint subsets of $A$ whose union is all of $A$
\end{define}

\textbf{Order Relations}

A relation $C$ in a set $A$ is called an \textbf{order relation} (or a \textbf{simple order} or a \textbf{linear order}) if it has the following properties:

\begin{itemize}
	\item (Comparability) For every $x$ and $y$ in $A$ for which $ x \neq y$, either $ xCy$ or $yCx$.
	\item (Nonreflexivity) For no $x$ in $A$ does the relation $xCx$ holds.
	\item (Transitivity) If $xCy$ and $yCz$, then $xCZ$.
\end{itemize}

\begin{define}
	If $X$ is a set and $,$ is an order relation $X$, and if $a < b$, we use the notation $(a,b)$ to denote the set:
	\[ \set{x | a < x < b} \]
	it is called an \textbf{open interval } in $X$. If this set is empty, we call $a$ the immediate predecessor of $b$, and we call $b$ the \textbf{immediate successor} of $a$
\end{define}

\begin{define}
	Suppose that $A$ and $B$ are two sets with the order relations $ <_A $ and $ <_B $ respectively. We say that $A$ and $B$ have the same \textbf{order type} if there is a bijective correspondence between them that preserves order; that is, if there exists a bijective function $\function{f}{A}{B}$ such that:
	\[ a_1 <_A a_2 \implies f(a_1) <_B f(a_2) \]
	
\end{define}

The following is very useful in Topology:

\begin{define}
	Suppose that $A$ and $B$ are two sets with order relations $ <_A $ and $ < _B $ respectively. Define an order relation $ <$ on $A \times B$ by defining:
	\[ a_1 \times b_1  < a_2 \times b_2 \]
	if $a_1 <_A a_2$, or if $a_1 = a_2 $ and $ b_1 <_B b_2$. It is called the \textbf{dictionary order relation} on $ A \times B$
\end{define}

Some terminology is defined, Supoose that $A$ is a set ordered by the realtion $<$. Let $A_0$ be a subset of $A$. We say that the element $b$ is the \textbf{largest element} of $A_0$ if $b \in A_0$ and if $ x \leq b$ for every $x \in A_0$. Similarly, we say that $a$ is the \textbf{smallest element}, if $a \in A_0$ and if $a \leq x$ for every $x \in A_0$.

We say that the subset $A_0$ of $A$ is \textbf{bounded above} if there is an element $b$ of $A$ such that $x \leq b$ for every $ x \in A_0$; the element $b$ is called an \textbf{upper bound } fo $A_0$. If the set of all upper bounds for $A_0$ has a smallest element, that element is called the \textbf{least upper bound}, or the \textbf{supremum}, of $A_0$. It is denoted by $ \sup A_0$; it may or may not belong to $A_0$. If it does, it is the largest element of $A_0$.
	Similarly, $A_0$ is \textbf{bounded below} if there is an element $a$ of $A$ such that $ a \leq x$ for every $x \in A_0 $; the element $a$ is called a \textbf{lower bound} for $A_0$. If the set of all lower bounds for $A_0$ has a largest element, that element is called the \textbf{greatest lower bound}, or the \textbf{infimum}, of $A_0$. It is denoted by $ \inf A_0$; it may or may not belong to $A_0$. If it does, it is the smallest element of $A_0$
	
\begin{define}
	An ordered set $A$ is said to have the \textbf{least upper bound property} if every nonempty subset $A_0$ of $A$ that is bounded above has a least upper bound. Analogously, the set $A$ is said to have the \textbf{greatest lower bound property} if every nonempty subset $A_0$ of $A$ that is bounded below has a greatest lower bound.
\end{define}


\textbf{The Integers and the Real Numbers}

\begin{define}
	A \textbf{binary operation} on a set $A$ is a function $f$ mapping $A \times A$ into $A$ 
\end{define}

\textbf{Assumption}

We assume there exists a set $ \RR $, called the set of \textbf{real numbers}, two binary operations $+$ and $ \cdot$ on $ \RR $ called the addition and multiplication operations, respectively, and an order relation $ <$ on $ \RR$, such that the following properties hold:

\textit{Algebraic Properties}
\begin{enumerate}
	\item $ (x + y ) + z = x + (y+z) $ 
	
	$ (x \cdot y) \cdot z = x \cdot(y \cdot z) $
	
	\item $ x + y = y + x $
	
	$ x \cdot y = y \cdot x $
	
	\item There exist a unique element of $\RR$ called \textbf{zero}, denoted by $0$, such that $ x+ 0 = x$ for all $x$ in $\RR$.
	
	There exists a unique element of $\RR$ called \textbf{one}, different from $0$ and denoted by $1$, such that $ x \cdot 1 = x$ for all $x \in \RR$
	\item For each $x$ in $\RR$, there exists a unique $y$ in $\RR$ such that $ x+ y = 0$
	
	For each $x$ in $\RR$ different from $0$, there exists a unique $y$ in $\RR$ such that $x\cdot y = 1$ 
	
	\item $ x \cdot (y +z) = (x \cdot y )+ (x \cdot z)$ for all $x,y,z \in \RR$
	
\end{enumerate}

\textit{A Mixed Algebraic and Order Property}

\begin{enumerate}
	\setcounter{enumi}{5}
	\item If $ x > y$, then $ x +z > y+z $.
	
	If $ x > y  $ and $ z > 0 $, then $ x \cdot z > y \cdot z $
\end{enumerate}

\textit{Order Properties}

\begin{enumerate}
	\setcounter{enumi}{6}
	\item The order relation $ < $ has the least upper bound property.
	\item If $ x < y $ , there exists an element $z$ such that $ x < z$ and $ z < y$
\end{enumerate}

We define a number $x$ to be \textbf{positive} if $  x > 0 $ and to be \textbf{negative} if $ x<0 $. We denote the positive reals via $\RR_+$.

Any set with two binary operations satisfying $(1)$ - $(5)$ is called a \textbf{field}; if the field has an order relation satisfying $ (6) $ is called an \textbf{ordered field}. Any set with an order relation satisfying $(7)$ and $(8)$ is called by topologist a \textbf{linear continuum}.

\begin{define}
	A subset $A$ of the real numbers is said to be \textbf{inductive} if it contains the number $1$, and if for every $x$ in $A$, the number $x+1$ is also in $A$. Let $\gothic{A}$ be the collection of all inductive subsets of $\RR$. Then the set $\ZZ_+$ of \textbf{positive integers} is defined by the equation:
	\[ \ZZ_+ = \bigcap_{A \in \gothic{A}} A \]
	
\end{define}
