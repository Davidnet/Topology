\section{Set Theory and Logic}




\begin{defn}
	A \textbf{rule of assignment} is a subset $r$ of the cartesian product $C \times D$ of two sets, having the property that each element of $C$ appears as the first coordinate of at most one ordered pair belonging to $r$
\end{defn}
An equivalent formulation is
\[ [ (c,d) \in r \qt{ and } (c,d') \in r ] \implies [d = d'] \]

\begin{defn}
	a \textbf{function } $f$ is a rule of assignment $r$, together with a set $B$ that contains the image set of $r$ . The domain $A$ of the rule $r$ is also called the \textbf{domain} of the function $f$; the image set of $r$ is also called the \textbf{image set} of $f$; and the set $b$ is called the \textbf{range} of $f$
\end{defn}

\begin{defn}
	If $\function{f}{A}{B}$ and if $A_0$ is a subset og $A$, we define the \textbf{restriction} of $f$ to $A_0$ to be the function mapping $A_0$ into $ B $ whose rule is:
	\[ \set{(a,f(a)) | a \in A_0} \]
	It is denoted by $ \restric{f}{A_0} $.
\end{defn}

\begin{defn}
	Given functions $\function{f}{A}{B}$ and $\function{g}{B}{C} $, we define the \textbf{composite} $ g \circ f$ of $f$ and $g$ as the function $ \function{g \circ f}{A}{C} $ defined by the equation $ (g \circ f)(a) = g(f(a)) $.
	Formally, $  g \circ f: A \rightarrow C $ is the function whose rule is:
	\[ \set{(a,c) | \qt{For some } b \in B, f(a)= b \qt{ and } g(b) = c} \]
\end{defn}
We take note that $ g \circ f$ is defined only when the range of $f$ equals the domain of $g$
\begin{defn}
	A function $ \function{f}{A}{B} $ is said to be \textbf{injective} (or \textbf{one-to-one}) if for each pair of distinct points of $A$, their images under $f$ are distinct. It is said to be \textbf{surjective} (or $f$ is said to map $A$ \textbf{onto} $B$) if every element of $B$ is the image of some element of $A$ under the function $f$, If $f$ is both injective and surjective, it is said to be \textbf{bijective} 
\end{defn}
An important remark of facts is that, the composite of two surjective functions is surjective, and the composite of two injective functions is injective.

If $f$ is bijective there exists a function from $B$ to $A$ called the \textbf{inverse} of $f$. It is denoted by $ f^{-1} $ and is defined by letting $ f\inv(b) $ be that unique element $a$ of $A$   for which $f(a) = b$. 

\begin{lem}
	Let $\function{f}{A}{B} $. If there are functions $ \function{g}{B}{A} $ and $ \function{h}{B}{A} $ such that $ g(f(a)) = a $ for every $a$ in $A$, and $f(h(b)) = b$ for every $b$ in $B$, then $f$ is bijective and $ g = h= f\inv $
\end{lem}
\begin{proof}
	To prove that $f$ is surjective, observe that the function $h$ that goes from $B$ to $A$ maps every element to $h(b) \in A$ so that, every element of $b$ has an element of $A$ such that $ f(h(b)) = b$. To prove that it is injective, suppose for the sake of contradiction that there exists elements of $A$ such that $ a \neq a' $ but $f(a) = f(a') $ then, applying $g$ to both sides of this equation we have that $ g(f(a)) = g(f(a')) $ and by the conditions of the hypothesis we have that $a = a'$ which is a contradiction. To see that both are equal observe that:
	\begin{align*}
	h(b) &= g(f(h(b))) \\
		 &= g(b)		
	\end{align*}
	and the proof is complete.
\end{proof}
\begin{defn}
	Let $ \function{f}{A}{B} $. If $A_0$ is a subset of $A$, we denote by $f(A_0)$ the set of all images of points of $A_0$ under the function $f$; this set is called the \textbf{image} of $A_0$ under $f$. Formally,
	\[ f(A_0) = \set{b | b = f(a) \qt{ for at least one } a \in A_0} \]
	On the other hand, if $ B_0 $  is a subset of $B$, we denote by $f\inv(B_0) $ the set of all elements of $A$ whose images under $f$ lie in $B_0$; it is called the \textbf{preimage} of $B_0$ under $f$ (or the ``counterimage'' or the ``inverse image'' of $B_0$). Formally,
	\[ f\inv(B_0) = \set{a | f(a) \in B_0} \]
	Of course, there may be no points $a$of $A$ whose images lie in $ B_0 $; in that case, $f\inv (B_0) $ is empty.
	
\end{defn}

We make the following remark, and that is that: If $ \function{f}{A}{B} $  and if $A_0 \contained A$ and $B_0 \contained B$, then:
\[ A_0 \contained f\inv(f(A_0)) \quad \qt{and} \quad f(f\recip(B_0)) \contained B_0 \]

The first inclusion is an equality if $f$ is injective, and the second inclusion is an equality if $f$ is surjective.

\textbf{Relations}

\begin{defn}
	A \textbf{relation} on a set $A$ is a subset $C$ of the Cartesian product $A \times A$ 
\end{defn}

\textbf{Equuivalence Relations and Partitions}
An \textbf{equivalence relation} on a set $A$ is a relation $C$ on $A$ having the following three properties:
\begin{enumerate}
	\item (Reflexivity) $xCx$ for every $ x \in A $
	\item (Symmetry) If $ xCy $, then $ yCx $ 
	\item (Transitivity) If $ xCy $ and $ yCz $, then $ xCz $
\end{enumerate}  

Given an equivalent realtion $ \sim $ on a set $A$ and an element $x$ of $A$, we define a certain subset $E$ of $A$, called the \textbf{equivalence class} determined by $x$, via:
\[ E = \set{y | y \sim x} \]

\begin{lem}
	Two equivalence classes $E$ and $E'$ are either disjoint or equal
\end{lem}

\begin{proof}
	Let $E$ be the equivalence class determined by $x$, and let $E'$ be the equivalence class determined by $x'$. Suppose that $ E \cap E'$ is not empty; let $y$ be a point of $ E \cap E'$, observe that we have $ y ~ x $ and $ y ~ x'$ so  that we conclude that $ x ~ x'$. If now $w$ is any point of $E$, we have $w ~ x$ so that we conclude $E \contained E'$, by symmetry of the argument we see that the equality holds.
\end{proof}

\begin{defn}
	A \textbf{partition} of a set $A$ is a collection of disjoint subsets of $A$ whose union is all of $A$
\end{defn}

\textbf{Order Relations}

A relation $C$ in a set $A$ is called an \textbf{order relation} (or a \textbf{simple order} or a \textbf{linear order}) if it has the following properties:

\begin{itemize}
	\item (Comparability) For every $x$ and $y$ in $A$ for which $ x \neq y$, either $ xCy$ or $yCx$.
	\item (Nonreflexivity) For no $x$ in $A$ does the relation $xCx$ holds.
	\item (Transitivity) If $xCy$ and $yCz$, then $xCZ$.
\end{itemize}

\begin{defn}
	If $X$ is a set and $,$ is an order relation $X$, and if $a < b$, we use the notation $(a,b)$ to denote the set:
	\[ \set{x | a < x < b} \]
	it is called an \textbf{open interval } in $X$. If this set is empty, we call $a$ the immediate predecessor of $b$, and we call $b$ the \textbf{immediate successor} of $a$
\end{defn}

\begin{defn}
	Suppose that $A$ and $B$ are two sets with the order relations $ <_A $ and $ <_B $ respectively. We say that $A$ and $B$ have the same \textbf{order type} if there is a bijective correspondence between them that preserves order; that is, if there exists a bijective function $\function{f}{A}{B}$ such that:
	\[ a_1 <_A a_2 \implies f(a_1) <_B f(a_2) \]
	
\end{defn}

The following is very useful in Topology:

\begin{defn}
	Suppose that $A$ and $B$ are two sets with order relations $ <_A $ and $ < _B $ respectively. Define an order relation $ <$ on $A \times B$ by defining:
	\[ a_1 \times b_1  < a_2 \times b_2 \]
	if $a_1 <_A a_2$, or if $a_1 = a_2 $ and $ b_1 <_B b_2$. It is called the \textbf{dictionary order relation} on $ A \times B$
\end{defn}

Some terminology is defined, Supoose that $A$ is a set ordered by the realtion $<$. Let $A_0$ be a subset of $A$. We say that the element $b$ is the \textbf{st element} of $A_0$ if $b \in A_0$ and if $ x \leq b$ for every $x \in A_0$. Similarly, we say that $a$ is the \textbf{smallest element}, if $a \in A_0$ and if $a \leq x$ for every $x \in A_0$.

We say that the subset $A_0$ of $A$ is \textbf{bounded above} if there is an element $b$ of $A$ such that $x \leq b$ for every $ x \in A_0$; the element $b$ is called an \textbf{upper bound } fo $A_0$. If the set of all upper bounds for $A_0$ has a smallest element, that element is called the \textbf{least upper bound}, or the \textbf{supremum}, of $A_0$. It is denoted by $ \sup A_0$; it may or may not belong to $A_0$. If it does, it is the st element of $A_0$.
	Similarly, $A_0$ is \textbf{bounded below} if there is an element $a$ of $A$ such that $ a \leq x$ for every $x \in A_0 $; the element $a$ is called a \textbf{lower bound} for $A_0$. If the set of all lower bounds for $A_0$ has a st element, that element is called the \textbf{greatest lower bound}, or the \textbf{infimum}, of $A_0$. It is denoted by $ \inf A_0$; it may or may not belong to $A_0$. If it does, it is the smallest element of $A_0$
	
\begin{defn}
	An ordered set $A$ is said to have the \textbf{least upper bound property} if every nonempty subset $A_0$ of $A$ that is bounded above has a least upper bound. Analogously, the set $A$ is said to have the \textbf{greatest lower bound property} if every nonempty subset $A_0$ of $A$ that is bounded below has a greatest lower bound.
\end{defn}


\textbf{The Integers and the Real Numbers}

\begin{defn}
	A \textbf{binary operation} on a set $A$ is a function $f$ mapping $A \times A$ into $A$ 
\end{defn}

\textbf{Assumption}

We assume there exists a set $ \RR $, called the set of \textbf{real numbers}, two binary operations $+$ and $ \cdot$ on $ \RR $ called the addition and multiplication operations, respectively, and an order relation $ <$ on $ \RR$, such that the following properties hold:

\textit{Algebraic Properties}
\begin{enumerate}
	\item $ (x + y ) + z = x + (y+z) $ 
	
	$ (x \cdot y) \cdot z = x \cdot(y \cdot z) $
	
	\item $ x + y = y + x $
	
	$ x \cdot y = y \cdot x $
	
	\item There exist a unique element of $\RR$ called \textbf{zero}, denoted by $0$, such that $ x+ 0 = x$ for all $x$ in $\RR$.
	
	There exists a unique element of $\RR$ called \textbf{one}, different from $0$ and denoted by $1$, such that $ x \cdot 1 = x$ for all $x \in \RR$
	\item For each $x$ in $\RR$, there exists a unique $y$ in $\RR$ such that $ x+ y = 0$
	
	For each $x$ in $\RR$ different from $0$, there exists a unique $y$ in $\RR$ such that $x\cdot y = 1$ 
	
	\item $ x \cdot (y +z) = (x \cdot y )+ (x \cdot z)$ for all $x,y,z \in \RR$
	
\end{enumerate}

\textit{A Mixed Algebraic and Order Property}

\begin{enumerate}
	\setcounter{enumi}{5}
	\item If $ x > y$, then $ x +z > y+z $.
	
	If $ x > y  $ and $ z > 0 $, then $ x \cdot z > y \cdot z $
\end{enumerate}

\textit{Order Properties}

\begin{enumerate}
	\setcounter{enumi}{6}
	\item The order relation $ < $ has the least upper bound property.
	\item If $ x < y $ , there exists an element $z$ such that $ x < z$ and $ z < y$
\end{enumerate}

We defn a number $x$ to be \textbf{positive} if $  x > 0 $ and to be \textbf{negative} if $ x<0 $. We denote the positive reals via $\RR_+$.

Any set with two binary operations satisfying $(1)$ - $(5)$ is called a \textbf{field}; if the field has an order relation satisfying $ (6) $ is called an \textbf{ordered field}. Any set with an order relation satisfying $(7)$ and $(8)$ is called by topologist a \textbf{linear continuum}.

\begin{defn}
	A subset $A$ of the real numbers is said to be \textbf{inductive} if it contains the number $1$, and if for every $x$ in $A$, the number $x+1$ is also in $A$. Let $\gothic{A}$ be the collection of all inductive subsets of $\RR$. Then the set $\ZZ_+$ of \textbf{positive integers} is defined by the equation:
	\[ \ZZ_+ = \bigcap_{A \in \gothic{A}} A \]
	
\end{defn}

The following are properties of: $ \ZZ_+ $ 

\begin{enumerate}
	\item $ \ZZ_+ $ is inductive.
	\item (Principle of Induction) If $A$ is an inductive set of positive integers then $ A= \ZZ_+ $
\end{enumerate}
We define the set $ \ZZ $ of \textbf{integers} to be the set consisting of the positive integers $ \ZZ_+ $ and the number $ 0  $. The set $ \QQ  $ of quotients of integers is called the set of \textbf{rational numbers}.
We define
\[ \set{1, \ldots, n} = S_{n+1} \]

\begin{thm}
	\textbf{Well-ordering property}. Every nonempty subset of $ \ZZ_+ $ has a smallest element.
\end{thm}

\begin{proof}
	We first prove that, for each $ n \in \ZZ_+ $, the following statement holds: Every nonempty subset of $ \set{1, \ldots, n} $ has a smallest element.
	By induction. Let $A$ be the set of all positive integers $ n $ for which this statement holds. Then $ A $ contains 1, then, supposing $ A $ contains $ n $, we show that it contains $ n+1 $. So let $ C $ be a nonempty subset of the set $ \set{1, \ldots, n} $. If $ C $ consists of only the element $ n+1 $ we are done, otherwise consider the set $ C \cap \set{1, \ldots n} $ which is nonempty. Because $ n \in A $, this set has a smallest element, which will automatically be the smallest element of $C$ also. Thus $A$ is inductive, and we conclude $ A=\ZZ_+ $. Now we prove the theorem. Suppose that $D$ is a nonempty subset of $\ZZ_+$. Choose an element $n$ of $D$. Then the set $A = D \cap \set{1,..,D}$ is nonempty, os that $A$ has a smallest element $k$. The element $k$ is automatically the smallest element of $D$ as well.
\end{proof}

\textbf{ \LARGE Cartesian Product}

\begin{defn}
	Let $ \gothic{A} $ be a nonempty collection of sets. An \textbf{indexing function } for $ \gothic{A} $ is a surjective function $f$ from some set $J$ , called the \textbf{index set}, to $ \gothic{A} $. The collection $ \gothic{A} $, together with the indexing function $f$, is called an \textbf{indexed family of sets}. Given $ \alpha \in J $, we shall denote the set $ f(\alpha) $ by the symbol $ A_\alpha $. And we shall denote the indexed family itself by the symbol:
	\[ \set{A_\alpha}_{\alpha \in J} \]
	
\end{defn}

The indexing function it is not required to be injective. Two especially useful index sets are the set $ \set{1, \ldots,n} $ of the positive integers to $n$ and the set $ \ZZ_+ $.

\begin{defn}
	Let $m$ be a positive integer. Given a set $X$, we define an \textbf{m-tuple } of elements of $X$ to be a function.
	\[ \textbf{x}: \set{1, \ldots, m} \rightarrow X \].
	If $\textbf{x}$ is m-tuple, we often denote the value of \textbf{x} at $i$ by the symbol $x_i$ and call it the ith \textbf{coordinate} of \textbf{x}. And we often denote the function \textbf{x} itself by the symbol:
	\[ (x_1, \ldots,x_m) \]
\end{defn}

Now let $ \set{A_1, \ldots A_m} $ be a family of sets indexed with the set $ \set{1, \ldots, m} $. Let $ X = A_1 \cup \ldots \cup A_m $. We define the \textbf{cartesian product} of this indexed family, denoted by:
\[ \prod_{i=1}^{m} A_1 \quad \qt{or} \quad A_1 \times \ldots \times A_m.  \]
to be the set of all m-tuples $(x_1, \ldots, x_m) $ of elements of $X$ such that $ x_i \in A_i $ for each $i$.

\begin{defn}
	Given a set $X$, we define an\textbf{ $ \omega$-tuple} of elements of $X$ to be a function:
	\[ \textbf{x}: \ZZ_+ \rightarrow X \]
	we also call such a function a \textbf{sequence}, or an \textbf{infinite sequence}, of elements of $X$. If $\textbf{x}$ is an $\omega$-tuple, we often denote $\textbf{x} $ by the symbol:
	\[ (x_1, \ldots) \quad \qt{or} \quad (x_n)){n\in \ZZ_+} \]
\end{defn}	
Now let $ \set{A_1, \ldots} $ be a family of sets, indexed with the positive integers; let $X$ be the union of the sets in this family. The \textbf{Cartesian product} of this indexed family of sets, denoted by:

 \[ \prod_{i \in \ZZ_+} A_i \quad \qt{or} \quad A_1 \times \ldots \] 
 
 is defined to be the set of all $ \omega $-tuples $ (x_1,X_2, \ldots)  $ of elements of $X$ such that $ x_i \in A_i $ for each $i$

\textbf{Examples}
If $ \RR $ is the set of real numbers, then $ R^m $ denotes the set of all $ m $-tuples of real numbers; it is often called \textbf{euclidean $ m $-space}. Analogously, $ R^\omega $ is sometimes called ``infinite-dimensional euclidean space''.

\textbf{\LARGE Finite sets}

\begin{defn}
	A set is said to be \textbf{finite} if there is a bijective correspondence of $A$ with some section of the positive integers. That is, $A$ is finite if it is empty or if there is a bijection:
	\[ \function{f}{A}{\set{1,\ldots,n}} \]
	for some positive integer $n$. In the former case, we say that $A$ has \textbf{cardinality $0$}; in the latter case, we say that $A$ has \textbf{cardinality $n$}
\end{defn}

\textbf{Lemma} Let $n$ be a positive integer. Let $A$ be a set; let $a_0$ be an element of $A$. Then there exists a bijective correspondence $f$ of the set $A$ with the set $\set{arg1}, \ldots n+1$ if and only if there exists a bijective correspondence $g$ of the set $ A - \set{a_0}$ with the set $ \set{1, \ldots, n} $.

From this lemma we derive some important results.

\begin{thm}
	Let$A$ be a set; suppose that there exists a bijection $ \function{f}{A}{\set{1, \ldots n}}$ for some $n \in \ZZ_+$. Let $B$ be a proper subset of $A$. Then there exists no bijection $\function{g}{B}{\set{1, \ldots n}}$; but (provided $B \neq \varnothing$) there does exist a bijection $\function{h}{B}{\set{1, \ldots,m}}$ for some $ m < n$
\end{thm}

\textbf{Corollary } If $ A $ is finite, there is no bijection of $A$ with a proper subset of itself.

\textit{Proof} Assume that $B$ is a proper subset of $A$ and that $ \function{f}{A}{B} $ is a bijection. By assumption, there is a bijection $ \function{g}{A}{\set{1, \ldots, n}} $ for some n. The composite $ g \circ f\inv  $ is then a bijection of $B$ with $ \set{1, \ldots, n} $. This contradicts the preceding theorem.

\textbf{Corollary}  The cardinality of a finite set $A$ is uniquely determined by $A$.

\textit{Proof} Let $ m<n $. Suppose there are bijections:

\begin{align*}
f: A &\rightarrow {\set{1, \ldots, n}} \\
g: A &\rightarrow \set{1, \ldots, m}
\end{align*}
Then the composite:
\[ g \circ f\inv: \set{1, \ldots,n} \rightarrow \set{1, \ldots,m} \]
is a bijection of the finite set $ \set{1, \ldots,n} $ with a proper subset of itself, contradicting the corollary just proved.

\textbf{Corollary} If $B$ is a subset of the finite set $A$, then $B$ is finite. IF $B$ is a proper subset of $A$, then the cardinality of $B$ is less than the cardinality of $A$.

\textbf{Corollary} Let $B$ be a nonempty set. Then the following are equivalent:
\begin{enumerate}
	\item $B$ is finite.
	\item There is a surjective function from a section of the positive integers onto $B$.
	\item There is an injective function from $B$ into a section of the positive integers.
\end{enumerate}

\textit{Proof} $ 1 \implies 2 $. Since $B$ is nonempty, there is, for some $n$, a bijective function $ \function{f}{\set{1, \ldots, n}}{B} $

$ 2 \implies 2 $ If $\function{f}{\set{1, \ldots, n}}{B}$ is surjective, define $ \function{g}{B}{\set{1, \ldots,n}} $ by the equation:
\[ g(b) = \qt{ smallest element of } f\inv(\set{b}) \]
Because $f$ is surjective, the set $ f\inv\set{(b)} $ is nonempty; then the well-ordering property of $ \ZZ_+ $ tell us that $ g(b) $ is uniquely defined. The map $g$ is injective, for if $ b \neq b'$ then the sets $ f\inv(\set{b}) $  and $ f\inv(\set{b'}) $

$ 3 \implies 1 $ If $ \function{g}{B}{\set{1, \ldots, n}} $ is injective, then changing the range of $g$ gives a bijection of $B$ with a subset of $ \set{1, \ldots, n} $. It follows from the preceding corollary that $B$ is finite. \qedsymbol

\textbf{Corollary} Finite unions and finite cartesian products of finite sets are finite.


\textbf{\LARGE Countable and Uncountable Sets}

\begin{defn}
	A set $A$ is said to be \textbf{infinite} if it is not finite. It is said to be \textbf{countably infinite} if there is a bijective correspondence:
	\[ \function{f}{A}{\ZZ_+} \]
	
\end{defn}

\begin{defn}
	A set is said to be \textbf{countable} if it is either finite or countably infinite. A set that is not countable is said to be \textbf{uncountable}
\end{defn}

a useful criterion for showing that a set is countable is the following:

\begin{thm} \label{countable criterion}
	Let $B$ be a nonempty set. Then the following are equivalent:
	\begin{enumerate}
		\item $B$  is countable.
		\item There is a surjective functions $ \function{f}{\ZZ_+}{B} $.
		\item There is an injective function $ \function{g}{B}{\ZZ_+} $
	\end{enumerate}
\end{thm}

\textit{Lemma } If $C$ is an infinite subset of $ \ZZ_+ $, then $C$ is countably infinite.

To avoid logical problems, we define then the following principle.

\textbf{Principle of recursive} Let $A$ be a set. Given a formula that defines $ h(1) $ as a unique element of $A$, and for $ i > 1 $ defines $ h(i) $ uniquely as an element of $A$ in terms of the values of $h$ for positive integers less than $i$, this formula determines a unique function $ \function{h}{\ZZ_+}{A} $

\textbf{Corollary } A subset of a countable set is countable.

\textit{Proof} Suppose $A \subset B$, where $B$ is countable. There is an injection $f$ of $B$ into $ \ZZ_+ $; the restriction of $f$ to $A$ is an injection of $A$ into $ \ZZ
_+ $

\textbf{Corollary } The set $ \ZZ_+ \times \ZZ_+ $ is countably infinite

\textit{Proof} In view of \ref{countable criterion}, it suffices to construct an injective map $ \function{f}{\ZZ_+ \times \ZZ_+}{\ZZ_+} $. We define $f$ by the equation:

\[ f(n,m) = 2^n\cdot 3^m \]

\emph{$f$ is injective}, for suppose that $ 2^n3^m = 2^p3^q $. If $ n<p $, then $ 3^m = 2^{p-n}3^q $,  contradicting the fact that $ 3^m $ is odd for all $ m $. Therefore, $ n = p $. as a result, $ 3^m = 3^q $, then if $ m < q $, it follows that $ 1 = 3^{q-m} $, another contradiction hence $ m = q $

\begin{thm}
	A countable union of countable sets is countable, also a finite product of countable sets is countable.
\end{thm}

Observe that is very tempting to assert that countable products of countable sets should be countable; but this assertion is in fact not true.

\begin{thm}
	Let $ X $ denote the two element set $ \set{0,1} $. Then the set $ X^{\omega} $ is uncountable.
\end{thm}

\textit{Proof} We show that, given any function:
\[ g: \ZZ_+ \rightarrow X^\omega \]
g is not surjective. For this purpose, let us denote $ g(n) $ as
\[ g(n) = (x_{n1},x_{n2},x_{n3}, \ldots, x_{nm}, \ldots) \]
where each $ x_{ij} $ is either $ 0 $ or $ 1 $. Then we define an element $ \textbf{y} = (y_1,y_2, \ldots) $ of $ X^\omega $ by letting:

\[ y_n = \begin{cases}
0 &\qt{ if } x_nn =1\\
1 &\qt{ if } x_nn =0\\
\end{cases} \]
Now $ \textbf{y} $ is an element of $ X^\omega $, and $ \textbf{y} $ does not lie in the image of $ g $; given $ n $, the point $ g(n) $ and the point $ \textbf{y} $ differ in at least one coordinate. \qed

\begin{thm}
	Let $ A $ be a set. There is no injective map $ \function{f}{\PP(A) }{A} $, and there is no surjective map $ \function{g}{A}{\PP(A)} $
\end{thm}

\textbf{\LARGE Infinte Sets and the Axiom of Choice}

\begin{thm}
	Let $A$ be a set. The following statements about $A$ are equivalent:
	\begin{enumerate}
		\item There exists an injective function $ \function{f}{\ZZ_+}{A} $
		\item There exists a bijection of $A$ with a proper subset of itself.
		\item $A$ is infinite.
	\end{enumerate}
\end{thm}

The proof of this theorem allows us to discuss an important method of forming sets. \textbf{The Axiom of Choice}

\textbf{Axiom of choice} Given a collection $ \mathcal{A} $ of disjoint nonempty sets, there exists a set $ C $ consisting of exactly one element from each element of $ \gothic{A} $; that is, a set $C$ such that $C$ is contained in the union of the elements of  $ \gothic{A} $, and for each $ A \in \gothic{A} $, the set $C \cap A$ contains a single element.

\textbf{Lemma (Existence of a choice function)} Given a collection $ \gothic{B} $ of nonempty sets (not necessarily disjoint), there exists a function:
\[ \function{c}{\gothic{B}}{\bigcup_{B \in \gothic{B}}B} \]
such that $ c(B) $ is an element of $B$, for each $b \in \gothic{B}$

\textbf{\LARGE Well-Ordered Sets}

\begin{defn}
	A set $A$ with an order relation $ < $ is said to be \textbf{well-ordered} if every nonempty subset $A$ has a smallest element.
\end{defn}