\subsection{Relations}
\begin{define}
	A \textbf{relation} on a set $ A $ is a subset $ C $ of the cartesian product $ A \times A $
\end{define}

\subsubsection*{Equivalence Relations and Partitions}

An \textbf{equivalence relation} on a set $ A $ is a relation $ C $ on A having the following three properties:
\begin{enumerate}
	\item (Reflexivity) $ xCx $
	\item (Symmetry) $ xCy $, then $ yCx $
	\item (Transitivity) $ xCy  $ and $ yCz \implies xCz $
\end{enumerate}

Given an equivalence relation $ \sim $ on a set $ A $ and an element $ x  $ of $ A $, we define a certain subset $ E $ of $ A $, called the \textbf{equivalence class} determined by $ x $, by the equation:
\[ E = \set{y | y \sim x} \]

\begin{lem}
	Two equivalence classes $ E $ and $ E' $ are either disjoint or equal.
\end{lem}

\begin{proof}
	Let $ E $ be the equivalence class determined by $ x $, and $ E' $ be the equivalence class determined by $ x' $, suppose $ E \cap E'  $ is not empty, so that $ y \in E \cap E' $, since $ y $ belongs to $ E $ and $ E' $, $ x \sim y $ and $ y \sim x' $ so that $ x \sim x' $ so that $ x' \in E $ and similarly $  x \in E' $ and we conclude that $ E = E' $
\end{proof}

\begin{defn}
	A \textbf{partition} of a set $ A $ is a collection of disjoint nonempty subsets of $ A $ whose union is all of $ A $
\end{defn}

\begin{lem}
	Given any partition $ \mathcal{D} $ of $ A $, there is exactly one equivalence relation on $ A $ from which is derived.
\end{lem}
\begin{proof}
	To show that the partition $ \mathcal{D} $ comes from some equivalence relation, let us define the equivalence relation on $ A $ as: let $ x \in A $ $ x \sim y $ if and only if $ x,y  $ belong to the same element $ \mathcal{D} $. Observe that this relation complies with all the rules to be a equivalence relation.
	Now for the uniqueness suppose $ \sim_1 $ and $ \sim_2 $ gives arise to the same partition $ \mathcal{D} $ to do that we will show that $ x \sim_1 y \iff x\sim_2y $ but that is simple, consider $ E $ as the equivalence class determined by $ x $ in $ \sim_1 $ and consider $ E' $ with $ \sim_2 $, since $ E_1 $ is an element of $ \mathcal{D} $ then it must equal the unique element $ D  $ of $ \mathcal{D} $, so that $ E = D =E' $
\end{proof}
\subsection*{Order Relations}
A relation $ C $ on a set $ A $ is called an \textbf{order relation} (or a \textbf{simple order} or a \textbf{linear order}) if it has the following properties:
\begin{enumerate}
	\item (Comparability) For every $ x $ and $ y $ in $ A $ for which $ x \neq y $ either $ xCy $ or $ yCx $
	\item (Non reflexivity) For no $ x $ in $ A $ does the relation $ xCx $ holds.
	\item (Transitivity) If $ xCy $ and $ yCz $, then $ xCz $
\end{enumerate}

\begin{defn}
	If $ X $ is a set and $ < $ is an order relation on $ X $, and if $ a < b $ we use the notation $ (a,b) $ to denote the set:
	\[ \set{x| a < x < b} \];
	it is called an \textbf{open interval } in $ X $. If this set is empty, we call $ a $ the \textbf{immediate predecessor } of $ b $ and $ b $ the \textbf{immediate successor} of $ a $
\end{defn}
\begin{defn}
	Suppose that $ A $ and $ B $ are two sets with order relations $ <_A $ and $ <_B $respectively. We say that $ A $ and $ B $ have the same \textbf{order type} if there is a bijective function $ f: A \rightarrow B $ such that:
	\[ a_1 <_A a_2 \implies f(a_1) <_B f(a_2) \]
\end{defn}
\begin{defn}
	Suppose that $ A $ and $ B $ are two sets with order relations $ <_A $ and $ <_B $ respectively. Define an order relation $ < $ on $ A \times B $ by defining:
	\[ a_1 \times b_1 < a_2 \times b_2 \]
	if $ a_1 < a_2 $, or if $ a_1 = a_2  $ and $ b_1 <_b b_2 $. It is called the \textbf{dictionary order relation} on $ A \times B $
\end{defn}
For the next definitions Let $ A $ be a set ordered by the relation$ < $
\begin{define}
	Let $ A_0 $ be a subset of $ A $. We say that the element $ b $ is the \textbf{largest element} of $ A_0 $ if $ b \in A_0 $ and if $ x \leq b $ for every $ x \in A_0 $
\end{define}
Similarly, we define:
\begin{defn}
	Let $ A_0 $ be a subset of $ A $. We say that the element $ a $ is the \textbf{smallest element} of $ A_0 $ if $ a \in A_0 $ and if $  a \leq x $ for every $ x \in A_0 $
\end{defn}
We make the remark that a set has at most one largest element and at most one smallest element.

\begin{define}
	We say the subset $ A_0 $ is \textbf{bounded above } if there is an element $ b $ of $ A $ such that $ x \leq b $ for every $ x \in A_0 $; the element $ b $ is called an \textbf{upper bound} for $ A_0 $. If the set of all upper bounds for $ A_0 $ has a smallest element, that element is called the \textbf{least upper bound}, or the \textbf{supremum} of $ A_0 $ often denoted as $ \sup A_0 $
\end{define}
We make the remark that the supremum may not be in $ A_0 $ if it is in $ A_0 $ then it is the \textbf{largest element} of $ A_0 $.
Similarly
\begin{define}
	We say the subset $ A_0 $ is \textbf{bounded below} if there is an element $ a $ of $ A $ such that $ a \leq x $ for every $ x \in A_0 $; the element $ a $ is called a \textbf{lower bound} for $ A_0 $. If the set of all lower bounds for $ A_0 $ has a largest element, that element is called the \textbf{greatest lower bound} or the \textbf{infimum} of $ A_0 $. It is denoted by $ \inf A_0  $
\end{define}
We may do the remark that $ \inf A_0 $ may not belong to $ A_0 $, if it does, then it is the $ \textbf{smallest element} $

Now, we can define the least upper bound property.

\begin{defn}
	An ordered set $ A $ is said to have the \textbf{least upper bound property} if every nonempty subset $ A_0 $ of $ A $ that is bounded above has a least upper bound. Analogously, the set $ A $ is sad to have the \textbf{greatest lower bound property} if every nonempty subset $ A_0 $ of $ A $ that is bounded below has a greatest lower bound
\end{defn}
\subsubsection{Exercises}
\begin{exc}
	Find the flaw in the following proof: suppose that $ C $ is a symmetric and transitive relation, then $ xCy \implies yCx $ and by transitive then $ xCx $ so that $ C $ is reflexive
\end{exc}
\begin{sol}
	This is true if either $ xCy $ or $ yCx $ if neither of them holds then, this does not prove that $ xCx $
\end{sol}
\begin{exc}
	Show that given any collection of equivalence relations on a set $ A $, their intersection is an equivalence relation on $ A $
\end{exc}
\begin{sol}
	Let $ S_i $ be a collection of equivalence relations on $ A $. consider $ S = \bigcup_{i \in I}S_i $:
	\begin{enumerate}
		\item Let $ x \in A $, since each of the $ S_i  $ is a reflexive relation $ xS_ix $ for each $ i \in I $, that is $ xSx $
		\item Let $ x,y \in S  $ so that $ xS_i y $ for all $ i \in I $, then since $ S_i $ is symmetric $ yS_i x \forall i \in I $ so that $ ySx $
		\item Consider $ x,y,z $ such that $ xSy $ and $ ySz $, that is: for all $ i \in I $ the following holds: $ xS_i y $ and $ y S_i z $ since each $ S_i  $ is transitive $ x S_i z $ for all  $ i $ so that $ xSz $
	\end{enumerate}
\end{sol}


