\subsection{Relations}
\begin{define}
	A \textbf{relation} on a set $ A $ is a subset $ C $ of the cartesian product $ A \times A $
\end{define}

\subsubsection*{Equivalence Relations and Partitions}

An \textbf{equivalence relation} on a set $ A $ is a relation $ C $ on A having the following three properties:
\begin{enumerate}
	\item (Reflexivity) $ xCx $
	\item (Symmetry) $ xCy $, then $ yCx $
	\item (Transitivity) $ xCy  $ and $ yCz \implies xCz $
\end{enumerate}

Given an equivalence relation $ \sim $ on a set $ A $ and an element $ x  $ of $ A $, we define a certain subset $ E $ of $ A $, called the \textbf{equivalence class} determined by $ x $, by the equation:
\[ E = \set{y | y \sim x} \]

\begin{lem}
	Two equivalence classes $ E $ and $ E' $ are either disjoint or equal.
\end{lem}

\begin{proof}
	Let $ E $ be the equivalence class determined by $ x $, and $ E' $ be the equivalence class determined by $ x' $, suppose $ E \cap E'  $ is not empty, so that $ y \in E \cap E' $, since $ y $ belongs to $ E $ and $ E' $, $ x \sim y $ and $ y \sim x' $ so that $ x \sim x' $ so that $ x' \in E $ and similarly $  x \in E' $ and we conclude that $ E = E' $
\end{proof}

\begin{defn}
	A \textbf{partition} of a set $ A $ is a collection of disjoint nonempty subsets of $ A $ whose union is all of $ A $
\end{defn}

\begin{lem}
	Given any partition $ \mathcal{D} $ of $ A $, there is exactly one equivalence relation on $ A $ from which is derived.
\end{lem}
\begin{proof}
	To show that the partition $ \mathcal{D} $ comes from some equivalence relation, let us define the equivalence relation on $ A $ as: let $ x \in A $ $ x \sim y $ if and only if $ x,y  $ belong to the same element $ \mathcal{D} $. Observe that this relation complies with all the rules to be a equivalence relation.
	Now for the uniqueness suppose $ \sim_1 $ and $ \sim_2 $ gives arise to the same partition $ \mathcal{D} $ to do that we will show that $ x \sim_1 y \iff x\sim_2y $ but that is simple, consider $ E $ as the equivalence class determined by $ x $ in $ \sim_1 $ and consider $ E' $ with $ \sim_2 $, since $ E_1 $ is an element of $ \mathcal{D} $ then it must equal the unique element $ D  $ of $ \mathcal{D} $, so that $ E = D =E' $
\end{proof}
\subsection*{Order Relations}
A relation $ C $ on a set $ A $ is called an \textbf{order relation} (or a \textbf{simple order} or a \textbf{linear order}) if it has the following properties:
\begin{enumerate}
	\item (Comparability) For every $ x $ and $ y $ in $ A $ for which $ x \neq y $ either $ xCy $ or $ yCx $
	\item (Non reflexivity) For no $ x $ in $ A $ does the relation $ xCx $ holds.
	\item (Transitivity) If $ xCy $ and $ yCz $, then $ xCz $
\end{enumerate}

\begin{defn}
	If $ X $ is a set and $ < $ is an order relation on $ X $, and if $ a < b $ we use the notation $ (a,b) $ to denote the set:
	\[ \set{x| a < x < b} \];
	it is called an \textbf{open interval } in $ X $. If this set is empty, we call $ a $ the \textbf{immediate predecessor } of $ b $ and $ b $ the \textbf{immediate successor} of $ a $
\end{defn}
\begin{defn}
	Suppose that $ A $ and $ B $ are two sets with order relations $ <_A $ and $ <_B $respectively. We say that $ A $ and $ B $ have the same \textbf{order type} if there is a bijective function $ f: A \rightarrow B $ such that:
	\[ a_1 <_A a_2 \implies f(a_1) <_B f(a_2) \]
\end{defn}
\begin{defn}
	Suppose that $ A $ and $ B $ are two sets with order relations $ <_A $ and $ <_B $ respectively. Define an order relation $ < $ on $ A \times B $ by defining:
	\[ a_1 \times b_1 < a_2 \times b_2 \]
	if $ a_1 < a_2 $, or if $ a_1 = a_2  $ and $ b_1 <_b b_2 $. It is called the \textbf{dictionary order relation} on $ A \times B $
\end{defn}
For the next definitions Let $ A $ be a set ordered by the relation$ < $
\begin{define}
	Let $ A_0 $ be a subset of $ A $. We say that the element $ b $ is the \textbf{largest element} of $ A_0 $ if $ b \in A_0 $ and if $ x \leq b $ for every $ x \in A_0 $
\end{define}
Similarly, we define:
\begin{defn}
	Let $ A_0 $ be a subset of $ A $. We say that the element $ a $ is the \textbf{smallest element} of $ A_0 $ if $ a \in A_0 $ and if $  a \leq x $ for every $ x \in A_0 $
\end{defn}
We make the remark that a set has at most one largest element and at most one smallest element.

\begin{define}
	We say the subset $ A_0 $ is \textbf{bounded above } if there is an element $ b $ of $ A $ such that $ x \leq b $ for every $ x \in A_0 $; the element $ b $ is called an \textbf{upper bound} for $ A_0 $. If the set of all upper bounds for $ A_0 $ has a smallest element, that element is called the \textbf{least upper bound}, or the \textbf{supremum} of $ A_0 $ often denoted as $ \sup A_0 $
\end{define}
We make the remark that the supremum may not be in $ A_0 $ if it is in $ A_0 $ then it is the \textbf{largest element} of $ A_0 $.
Similarly
\begin{define}
	We say the subset $ A_0 $ is \textbf{bounded below} if there is an element $ a $ of $ A $ such that $ a \leq x $ for every $ x \in A_0 $; the element $ a $ is called a \textbf{lower bound} for $ A_0 $. If the set of all lower bounds for $ A_0 $ has a largest element, that element is called the \textbf{greatest lower bound} or the \textbf{infimum} of $ A_0 $. It is denoted by $ \inf A_0  $
\end{define}
We may do the remark that $ \inf A_0 $ may not belong to $ A_0 $, if it does, then it is the $ \textbf{smallest element} $

Now, we can define the least upper bound property.

\begin{defn}
	An ordered set $ A $ is said to have the \textbf{least upper bound property} if every nonempty subset $ A_0 $ of $ A $ that is bounded above has a least upper bound. Analogously, the set $ A $ is sad to have the \textbf{greatest lower bound property} if every nonempty subset $ A_0 $ of $ A $ that is bounded below has a greatest lower bound
\end{defn}
\subsubsection{Exercises}
\begin{exc}
	Find the flaw in the following proof: suppose that $ C $ is a symmetric and transitive relation, then $ xCy \implies yCx $ and by transitive then $ xCx $ so that $ C $ is reflexive
\end{exc}
\begin{sol}
	This is true if either $ xCy $ or $ yCx $ if neither of them holds then, this does not prove that $ xCx $
\end{sol}
\begin{exc}
	Show that given any collection of equivalence relations on a set $ A $, their intersection is an equivalence relation on $ A $
\end{exc}
\begin{sol}
	Let $ S_i $ be a collection of equivalence relations on $ A $. consider $ S = \bigcup_{i \in I}S_i $:
	\begin{enumerate}
		\item Let $ x \in A $, since each of the $ S_i  $ is a reflexive relation $ xS_ix $ for each $ i \in I $, that is $ xSx $
		\item Let $ x,y \in S  $ so that $ xS_i y $ for all $ i \in I $, then since $ S_i $ is symmetric $ yS_i x \forall i \in I $ so that $ ySx $
		\item Consider $ x,y,z $ such that $ xSy $ and $ ySz $, that is: for all $ i \in I $ the following holds: $ xS_i y $ and $ y S_i z $ since each $ S_i  $ is transitive $ x S_i z $ for all  $ i $ so that $ xSz $
	\end{enumerate}
\end{sol}
\begin{exc}
	Show that an element in an ordered set has at most one immediate successor and at most one immediate predecessor. Show that a subset of an ordered set has at most one smallest element and at most one largest element
\end{exc}
\begin{sol}
	Consider an element $ a  \in A$, and $ A $ is an ordered set, consider $ a < b $, and $ a < b' $ for which:
	\[ \set{x | a <x < b} = \varnothing \]
	and
	\[ \set{x | a < x < b'} = \varnothing \]
	that is, both $ b $ and $ b' $ are the immediate successor of $ a $, then observe that if $ b < b' $, b will be in the first set, similarly $ b' $ will be in the second set, we conclude then that $ b = b' $.
	
	For the second question, arguments similar as above holds. 
\end{sol}
\begin{exc}
	Prove the following:
	\begin{thm}
		If an ordered set $ A $ has the least upper bound property, then it has the greatest lower bound property.
	\end{thm}
\end{exc}
\begin{sol}
	\begin{proof}
		Consider $ B $ subset of $ A $, and $ B $ is bounded below, consider the set of all lower bounds for $ B $, since $ B  $ is bounded below the above set is nonempty, and bounded above with every element of $ B $ since it is bounded above, it has a supremum, so that we have shown that the set of all lower bounds has a supremum, that is $ B $ has an infimum. Therefore $ A $ has the greatest lowest bound property.
	\end{proof}
	
\end{sol}

\begin{exc}
	If $ C $ is a relation on a set $ A $, define a new relation $ D $ on $ A $ by letting $ (b,a) \in D $ if $ (a,b) \in C $
	\begin{itemize}
		\item show that $ C $ is symmetric if and only if $ C = D $.
		\item Show that if $ C $ is an order relation, $ D $ is also an order relation.
		\item Prove the converse of the theorem in Exercise 13.
	\end{itemize}
\end{exc}
\begin{sol}
	\begin{itemize}
		\item Suppose $ C $ is symmetric, consider $ xDy $ or in other words $ yCx $, since $ C $ is symmetric, we have that $ xCy $, so that $ yDx $, and $ D $ is symmetric. Similar argument above for the case of $ D $ being the symmetric relation is similar.
		\item Suppose $ C $ is an order relation, since $ C $ is an order, then either $ yDx $ or $ xDy $ holds, similar for non-reflexivity, and transitivity.
	\end{itemize}
	For the converse of the above theorem.
	
\end{sol}

\section{The Integers and the Real Numbers}

\begin{defn}
	A \textbf{binary opertation} on a set $ A $ is a function $ f $ mapping $ A \times A $ into $ A $
\end{defn}

We assume there exists a set $ \RR $ called the set of \textbf{real numbers}, two binary operations $ + $ and $ \cdot $ on $ \RR $, called the addition and multiplication operations, respectively, and on order relation $ < $ on $ \RR $, such that the following properties hold:

\textit{Algebraic Properties}

\begin{enumerate}
	\item $ (x + y) + z = x + (y + z) $
	
	$ (x \cdot y) \cdot z = x \cdot (y \cdot z) $ for all $ x,y,z $ in $ \RR $
	
	\item $ x + y = y + x $,
	
	$ x \cdot y = y \cdot x $ for all $ x,y  $ in $ \RR $
	
	\item There exists a unique element of $ \RR $ called \textbf{zero}, denoted by $ 0 $, such that $ x + 0 = x $ for all $ x \in \RR $.
	
	There exists a unique element of $ \RR $ called \textbf{one}, different from $ 0 $ and denoted by $ 1 $, such that $ x \cdot 1 = x $ for all $ x \in \RR $.
	
	\item For each $ x $ in $ \RR $, there exist a unique $ y  $ in $ \RR $ such that $ x + y = 0 $.
	
	For each $ x $ in $ \RR $ different from $ 0 $, there exists a unique $ y $ in $ \RR $ such that $ x \cdot y = 1 $.
	
	\item $ x \cdot (y + z) = (x \cdot y) + (x \cdot z) $ for all $ x,y,z \in \RR $
\end{enumerate}

A \textit{Mixed algebraic and Order Property}.

\begin{enumerate}
	\setcounter{enumi}{5}
	\item if $ x > y $, then $ x +z > y + z $
	\item $ x > y $ and $ z > 0 $, then $ x\cdot z > y \cdot z $
\end{enumerate}

\textit{Order Properties}

\begin{enumerate}
	\setcounter{enumi}{7}
	\item The order relation $ < $ has the least upper bound property.
	\item If $ x < y $, there exists an element $ z $ such that $ x < z $ and $ z < y $
\end{enumerate}

If a set $ A $ with operations $ +, \cdot $ holds the (1) - (5) properties, then it is a \textbf{Field}, if it also has an order relation that holds (6) then it is an ordered field, if this order also gold (7) and (8) hen it is a \textbf{linear continuum}.

\begin{defn}
	A subset $ A $ of the real numbers is said to be \textbf{inductive} if it contains the number $ 1 $, and if for every $ x  $ in $ A $, the number $ x + 1 $ is also in $ A $. Let $ \mathcal{A} $ be the collection of all inductive subsets of $ \RR $. Then the set $ \ZZ_+ $ of \textbf{positive integers} is defined by the equation:
	\[ \ZZ_+ = \bigcap_{A \in \mathcal{A}} A \]
\end{defn}
We observe that the set $ \RR_+ $ is inductive, since it has the element $ 1 $ and the property that $  x > 0 $ then $ x + 1 > 0 $, therefore $ \ZZ_+ \subset \RR_+ $
Basic properties of $ \ZZ_+ $ are:
\begin{itemize}
	\item $ \ZZ_+ $ is inductive.
	\item (Principle of Induction). If $ A $ is an inductive set of positive integers, then $ A = \ZZ_+ $
\end{itemize}
We define the set $ \ZZ $ of \textbf{integers} to be the set consisting of the positive integers and the number $ 0 $, and the negatives of the elements $ \ZZ_+ $. If $ n $ is a positive integer, we use the symbol $ S_n $ to denote the set of all positive integers less than $ n $; we call it a \textbf{section} of the positive integers. $ S_1 $ is empty, and $ S_{n+1} $ denotes the set of positive integers between $ 1 $ and $ n $, inclusive.

\[ \set{1, \ldots, n} = S_{n+1} \]

\begin{thm}[Well-ordering property]
	Every nonempty subset of $ \ZZ_+ $ has a smallest element.
\end{thm}
\begin{proof}
	We first prove that, for each $ n \in \ZZ_+ $ the following statement holds: \textit{Every nonempty subset of $ \set{1, \ldots, n} $ has a smallest element}.
	Let $ A $ be the set of all positive integers $ n $ for which this statement holds, Then $ A $ contains $ 1 $, since if $ n = 1 $, the only nonempty subset of $ \set{1,\ldots,n} $ is the set $ 1 $ itself. Then supposing $ A $ contains $ n $, we show that contains $ n+1 $. So let $ C $ be a nonempty subset of $ \set{1, \ldots, n +1 } $. If $ C $ consists of the single element $ n+1 $ then that element is the smallest element of $ C $. Otherwise, consider the set $ C \cap \set{1, \ldots,n} $ which is nonempty. Because $ n \in A $, this set has a smallest element, which will automatically be the smallest element of $ C $ also. Thus, $ A $ is inductive, so we conclude that $  A = \ZZ_+ $; hence the statement is true for all $ n \in \ZZ_+ $.
	
	Now we prove the theorem. Suppose that $ D $ is a nonempty subset of $ \ZZ_+ $. Choose an element $ n $ of $ D $. Then the set $ D \cap \set{1, \ldots, n} $ is nonempty, so that $ A $ has a smallest element $ k $. Then $ k $ is automatically the smallest element of $ D $ as well.
\end{proof}

\begin{thm}[Strong induction principle]
	Let $ A $ be a set of positive integers. Suppose that for each positive integer $ n $, the statement $ S_n \cap A $ implies the statement $ n \in A $. Then $ A = \ZZ_+ $
\end{thm}

\begin{proof}
	If $ A $ does not equal all  of $ \ZZ_+ $, let $ n $ be the smallest positive integer that is not in $ A $. Then every positive integer less than $ n $ is in $ A $, so that $ S_n \subset A $. Our hypothesis implies that $ n \in A $, contrary to assumption.
\end{proof}

\subsubsection*{Exercise}
\begin{exc}
	Prove the following ``laws of algebra'' for: $ \RR  $.
	\begin{itemize}
		\item $ 0 \cdot x = 0 $
	\end{itemize}
\end{exc}
\begin{sol}
	\begin{itemize}
		\item Using hint:
		\begin{align*}
		x\cdot x = (x + 0) \cdot x \\
		x\cdot x = x \cdot x + 0 \cdot x
		\end{align*}
		by the property that $ x + y = x \implies y = 0 $ we conclude $ 0 \cdot x = 0 $
	\end{itemize}
\end{sol}
\begin{exc}
	\begin{itemize}
		\item Show that if $ \mathcal{A} $ is a collection of inductive sets, then the intersection of the elements of $ \mathcal{A} $ is an inductive set.
		\item Prove 
		\begin{itemize}
			\item $ \ZZ_+ $ is inductive.
			\item (Principle of Induction). If $ A $ is an inductive set of positive integers, then $ A = \ZZ_+ $
		\end{itemize}
	\end{itemize}
\end{exc}
\begin{sol}
	\begin{itemize}
		\item Suppose that $ \mathcal{A} $ is a set of inductive sets, observe that since an inductive set has the element $ 1 $, then $ 1 $ is in the intersection of these sets, now suppose $ x  $ belongs to the intersection of the elements of $ \mathcal{A} $, so that $ x $ is in every of the inductive sets, since all of them are inductive, $ x+1 $ is also in all the sets, therefore in the intersection of $ \mathcal{A} $.
		\item In a case by case scenario:
		\begin{itemize}
			\item By the above point $ \ZZ_+ $ is inductive.
			\item Suppose that $ A $ is an inductive set of positive integers, then $ A \subset \ZZ $, and by being inductive $ Z \subset A $
		\end{itemize}
	\end{itemize}
\end{sol}
\begin{exc}
	\begin{itemize}
		\item Prove by induction that given $ n \in \ZZ_+$, every nonempty subset of $ \set{1, \ldots, n} $ has a largest element.
		\item Explain why you cannot conclude from the above that every nonempty subset of $ \ZZ_+ $ has a largest element.
		
	\end{itemize}
\end{exc}
\begin{sol}
	\item Let $ A $ be the set of all positive integers for which this statement holds. $ A $ contains $ 1 $ since the set $ \set{1} $ has only one subset itself, and it contains the largest element. Now consider $ C $ subset of $ \set{1, \ldots, n+1} $, if $ n+1 \in C $ , then $ C $ has already its largest element, now consider $ n + 1 \notin C$, so that $ C $ is a subset of $ \set{1,\ldots,n} $ and by induction hypothesis it has a largest element.
	\item We are assuming that the set is contained in a set of the form $ \set{1, \ldots, n} $, but there are sets that are not contained in sets of that form.
\end{sol}
\begin{exc}
	\begin{itemize}
		\item Show that $ \RR $ has the greatest lower bound property.
	\end{itemize}
\end{exc}
\begin{sol}
	\begin{itemize}
		\item Consider $ A $ subset of $ \RR $ that is bounded above, consider the set $ L = \set{x | x \geq a, \forall a \in A} $
	\end{itemize}
\end{sol}