\textbf{Definición} $X$ es conexo por camino si $ \forall x,y \in X \exists f: [0,1] \rightarrow X$ continua tal que $ f(0) = x $ y $ f(1) = y $.

Conexo por caminos $ \implies $ conexo. Pero observar que conexo $ \not\implies $ conexo por caminos. El ejemplo siempre es \textbf{el seno del topologico}

Existe tambien la noción de \textbf{arcoconexo} si $X$ es arconexo por camino si $ \forall x,y \in X \exists f: [0,1] \rightarrow X$ continua tal que $ f(0) = x $ y $ f(1) = y $ y $f$ es homeomorfismo sobre su imagen.

\textbf{En un espacio de Hausdorff} $X$ es arcoconexo lo mismo que conexo por caminos.

\textbf{Definición} $X$ un espacio toplologico. $ \sim_c $, $ \sim_{cc} $, definimos dos relaciones de equivalencia $ x \sim_c y  $ si $ \exists A \overset{\qt{ conexo}}{\subseteq} X $ y $ x,y \in A $. $ x \sim_{cc} $ si $ \exists f_{\qt{ camino}} $ que me une $x$ a $y$

Las clases de equivalencia para la primera.\textbf{ Condicio de canlla contable}.

\textbf{Definición} $ B_c =\set{U \contain X | U \qt{ abierto y conexo}} $ y $ B_cc = \set{U \contain x | U \qt{ abierto y conexo por caminos}} $. Observar $ B_c \contain \tau  $ y $ B_cc \contain B_c $

$ X $ es \textbf{localmente conexo} si $ B_c $ es base para $ \tau $

$ X $ es localmente conexo por continuos si $ B_cc $ es base para $ \tau $

\textbf{Localmente conexo } no implica \textbf{localmente conexo por caminos}

\url{https://simomaths.wordpress.com/2013/03/10/topology-locally-connected-and-locally-path-connected-spaces/}